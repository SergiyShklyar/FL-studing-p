\documentclass{beamer}
\usepackage{amsmath}
\DeclareMathOperator{\ME}{\mathsf{E}}
\DeclareMathOperator{\var}{\mathsf{var}}
\DeclareMathOperator{\ceil}{\mathsf{ceil}}
\usepackage{amsthm}
\theoremstyle{plain}
\newtheorem{proposition}{Proposition}
\theoremstyle{definition}
\newtheorem{defn}{Definition}
\newtheorem{Gen2}{Generalization 2}
\theoremstyle{remark}
\newtheorem{remark}{Remark}
\title{Generalization of the fractional Brownian motion}
\subtitle{Gaussian Volterra processes with power-type kernels}
\author{Yuliya Mishura \and Sergiy Shklyar}
\begin{document}
\begin{frame}
	\titlepage
\end{frame}
\section{Motivation}
\begin{frame}
	\frametitle{Motivation}
	\framesubtitle{Fractional Brownian motion}
The fractional Brownian motion
\begin{itemize}
	\item $B^H$ --- Gaussian process; $\ME B^H = 0$;
	\item $B^H_0=0$,  $\ME(X_t-X_s)^2 = |t-s|^{2H}$.
\end{itemize}
$B^H$ is self-similar, has stationary increments.
$\ME B^H_t B^H_s = \frac{1}{2} \left(|t-s|^{2H}
- |t|^{2H} - |s|^{2H}\right)$.
It admits Molchan representation on $[0,T]$:
\begin{gather*}
B^H_t = \int_0^t K(t,s) \, dW_s, \\
%	\begin{aligned}
		K(t,s) = c_H s^{\frac12-H} \biggl( t^{H-\frac12} (t{-}s)^{H-\frac12}
		- 
\left(H{-}{\textstyle\frac12}\right) \int_s^t u^{H-\frac32} (u{-}s)^{H-\frac12}\, du
		\biggr)  %\quad \mbox{if $0<H<1$},
%	\end{aligned}
\end{gather*}
if $0<H<1$;
\begin{gather*}
K(t,s) = 
\left(H{-}{\textstyle\frac12}\right) c_H
s^{0.5-H} \int_s^t u^{H-0.5} (u-s)^{H-1.5} \, du \quad
\mbox{if $\frac12{<}H{<}1$},
\\
c_H = \left( \frac
{2 H \, \Gamma(1.5-H)}
{\Gamma(H+0.5)\, \Gamma(2-2H)}
\right)^{1/2} .
\end{gather*}
\end{frame}

\section{Generalization 1: Three functions}
\begin{frame}
	\frametitle{Generalization 1}
	\framesubtitle{Three functions}
	Consider the kernel $K(t,s)$ and the Volterra process $X$
\begin{gather}
K(t,s) = a(s) \int_s^t b(u) c(u-s) \, du,
\\
\label{eq:3Xabc}
X_t = \int_0^t a(s) \int_s^t b(u) c(u-s)\, du \, dW_s .
\end{gather}
\begin{proposition}\label{prop:1}
Let $a\in L^p[0,T]$, \, $b\in L^q[0,T]$
and $c\in L^r[0,T]$,
\addtocounter{equation}{1}
	\centerline{$\displaystyle
p\ge 2,\quad
q\ge 1,\quad
r\ge 1,\quad
\frac{1}{p}+\frac{1}{q}+\frac{1}{r}\le \frac32.
	$}\llap{(\theequation)}
	\begin{enumerate}
		\item
Then the process $X$ is well-defined on $[0,T]$.
\item
If $\frac{1}{p}+\frac{1}{r} < \frac32$,
then the process $X$ is  path-continuous.
\item
If $q>1$ and $\frac{1}{p}+\frac{1}{q}+\frac{1}{r} < \frac32$,
then the process $X$ is H\"older-continuous up
to order $\frac32 - 1/q - \max(\frac12, \, 1/p+1/r)$.
	\end{enumerate}
\end{proposition}
%Fractional Brownian motion with Hurst index $H>0.5$
%is a particular case of the fractional of process \eqref{eq:3Xabc}.
%However, Proposition~\ref{prop:1}
%underestimates the order in H\"older condition for the fBm.
\end{frame}

\begin{frame}
\begin{remark}
	\begin{gather}
	B^H_t = 
\int_0^t \left(H{-}{\textstyle\frac12}\right) c_H
s^{0.5-H} \int_s^t u^{H-0.5} (u-s)^{H-1.5} \, du \, dW_s,
		\nonumber\\
X_t = \int_0^t a(s) \int_s^t b(u) c(u-s)\, du \, dW_s .
		\tag{\ref{eq:3Xabc}}	
	\end{gather}
Proposition~\ref{prop:1}
underestimates the order in H\"older condition for the fBm
with Hurst index $H\in\bigl(\frac12,1\bigr)$.
\end{remark}
\begin{Gen2}<2>
\begin{equation}
X_t = \int_0^t s^\beta \int_s^t u^\beta (u-s)^\gamma \,  du \, dW_t.
\tag{\ref{eq:Xpower}}
\end{equation}
\end{Gen2}
\end{frame}


\section{Generalization 2: Three power functions}
\begin{frame}
	\frametitle{Generalization 2}
	\framesubtitle{Three power functions}
Consider the case where $a$, $b$ and $c$ are power functions.
\begin{equation}\label{eq:Xpower}
X_t = \int_0^t s^\beta \int_s^t u^\beta (u-s)^\gamma \,  du \, dW_t.
\end{equation}
The process $X_t$ is well-defined if
\begin{equation}\label{neq:condXpower}
\alpha>-\frac12, \quad \gamma>-1, \quad
\mbox{and} \quad
\alpha+\beta+\gamma>-\frac32.
\end{equation}
Properties: The process $X$ is
\begin{enumerate}
\item self-silimar with exponent
$H = \alpha+\beta+\gamma+\frac32 ,$
\item path-continuous,
\item H\"older-continuous up to order
up to order $\min(\gamma+\frac32, \, 1)$
on the interval $[t_0, T]$,
and

up to order $\min(H, \gamma+\frac32, \, 1)$
on the interval $[0, T]$.

Here $0<t_0<T<\infty$.
\item
If $\gamma>-\frac12$, the process $X$ is
$\ceil\bigl(\gamma+\frac12\bigr)$ times
continuously differentiable on $(0,+\infty)$.
\end{enumerate}
\end{frame}

\subsection{Incremental variance and generalized quasi-helix condition}
\begin{frame}
	\frametitle{Incremental variance}
\begin{proposition}\label{prop:localvar}
Let $t_0 > 0$.
The asymptotics of the incremental variance as $t\to t_0$,
$t_0>0$,
is as follows:
\begin{equation*}
% \refstepcounter{equation}
% \label{eq:asymp-localvar-1}
\begin{array}{l@{\qquad}l}
\displaystyle \var(X_{t}-X_{t_0}) \sim \frac
{t_0^{2\alpha+2\beta} \, |t - t_0|^{2\gamma+3} \,
\mathrm{B}(\gamma{+}1,\, {-}2\gamma{-}1)} {(\gamma + 1)\,(2\gamma
+ 3)}
& \mbox{if $\gamma<-\frac12$,}
\quad (\theequation) \\
\displaystyle \var(X_{t}-X_{t_0}) \sim t_0^{2\alpha+2\beta} \,
(t - t_0)^2 \, \ln\!\left( \frac{t_0}{|t - t_0|} \right)
& \mbox{if $\gamma=-\frac12$,} \\
\displaystyle \var(X_{t}-X_{t_0}) \sim
t_0^{2\alpha+2\beta+2\gamma+1} \, (t - t_0)^2 \,
\mathrm{B}(2\alpha{+}1,\, 2\gamma{+}1) & \mbox{if
$\gamma>-\frac12$.}
\end{array}
\end{equation*}
\end{proposition}
\begin{defn}<2>
Process $\{X_t, \; t\mathbin{\in}[t_0,T]\}$
is a \textit{generalized  quasi-helix\/}  with
exponents $\lambda_i>0, i=1,2$   if there exist two constants $C_i>0, i=1,2$ such that for any
$t_0  \le  t_1 < t_2  \le  T$
\[
C_1 (t_2-t_1)^{2\lambda_1} \le \var(X_{t_2} - X_{t_1}) \le
C_2 (t_2-t_1)^{2\lambda_2}.
\]
\end{defn}

\end{frame}

\begin{frame}

\vspace{3pt}{\def\arraystretch{1.5}
The exponents in the generalized quasi-helix condition:\\
\begin{tabular}{|p{0.42\linewidth}|p{0.27\linewidth}|p{0.2\linewidth}|}
\hline
& \multicolumn{2}{p{0.50\linewidth}|}{the exponents in the generalized  quasi-helix condition}\\
& on the interval $[0,T]$ are
& on any interval $[t_0,T]$ are\\
 \hline
If $\gamma\mathbin{<}-\frac12$ and $\alpha+\beta\le 0$,
& $\gamma{+}\frac32$ and $\alpha{+}\beta{+}\gamma{+}\frac32$
& $\gamma{+}\frac32$ \\ \hline
If $\gamma\mathbin{<}-\frac12$ and $\alpha+\beta\ge 0$,
& $\alpha{+}\beta{+}\gamma{+}\frac32$ and $\gamma{+}\frac32$
& $\gamma{+}\frac32$ \\ \hline
If $\gamma\mathbin{=}-\frac12$ and $\alpha+\beta< 0$,
& $1$ and $\alpha+\beta+1$
& $1$ and $1{-}\epsilon$ \\ \hline
If $\gamma\mathbin{=}-\frac12$ and $\alpha+\beta \ge 0$,
& $\alpha{+}\beta{+}1$ and $1{-}\epsilon$
& $1$ and $1{-}\epsilon$ \\  \hline
If $\gamma\mathbin{>}-\frac12$ and $\alpha{+}\beta{+}\gamma \mathbin{\le} -\frac12$,
& $1$ and $\alpha{+}\beta{+}\gamma{+}\frac32$
& $1$\\  \hline
If $\gamma\mathbin{>}-\frac12$ and $\alpha{+}\beta{+}\gamma \mathbin{\ge} -\frac12$,
& $\alpha{+}\beta{+}\gamma{+}\frac32$ and $1$
& $1$\\
\hline
\end{tabular}}
\end{frame}


\begin{frame}
	\frametitle{Generalization 2}
	\framesubtitle{Three power functions}
%Consider the case where $a$, $b$ and $c$ are power functions.
\begin{equation}\tag{\ref{eq:Xpower}}
X_t = \int_0^t s^\beta \int_s^t u^\beta (u-s)^\gamma \,  du \, dW_t.
\end{equation}
The process $X_t$ is well-defined if
\begin{equation}\tag{\ref{neq:condXpower}}
\alpha>-\frac12, \quad \gamma>-1, \quad
\mbox{and} \quad
\alpha+\beta+\gamma>-\frac32.
\end{equation}
Properties: The process $X$ is
\begin{enumerate}
\item self-silimar with exponent
$H = \alpha+\beta+\gamma+\frac32 ,$
\item<0> path-continuous,
\item H\"older-continuous up to order
up to order $\min(\gamma+\frac32, \, 1)$
on the interval $[t_0, T]$,
and

up to order $\min(H, \gamma+\frac32, \, 1)$
on the interval $[0, T]$.

Here $0<t_0<T<\infty$.
\item<0>
If $\gamma>-\frac12$, the process $X$ is
$\ceil\bigl(\gamma+\frac12\bigr)$ times
continuously differentiable on $(0,+\infty)$.
\end{enumerate}
\end{frame}

\begin{frame}
\frametitle{The inversere presentation}
\framesubtitle{Case $\gamma<0$}
Let
\[
\alpha>-\frac12, \quad -1<\gamma<0,
\quad \mbox{and}
\quad \alpha+\beta+\gamma > -\frac32.
\]
Then the inverse representation to \eqref{eq:Xpower}
is
$\displaystyle W_s = \int_0^t L(s,t) \, dX_t$
with
\begin{align*}
L(s,t) &=
- \frac{t^{-\beta}}{\mathrm{B}(\gamma{+}1, {-}\gamma)} \,
\frac{\partial}{\partial t}\!
\left( \int_t^s s^{-\alpha} (s-t)^{-\gamma-1} \, ds \right)
\\ &=
 \frac{t^{-\beta}}{\mathrm{B}(-\gamma, \gamma{+}1)}
\left(s^{-\alpha} (s{-}t)^{-\gamma-1}
+ \alpha \int_t^s u^{-\alpha-1} (u{-}t)^{-\gamma-1} \,du \right).
\end{align*}
\end{frame}

\begin{frame}
\frametitle{Differentiability}
If $\displaystyle\alpha>-\frac12$,\quad $\displaystyle\gamma>-\frac12$, and
$\displaystyle\alpha+\beta+\gamma>-\frac32$, 
then
the process $X$ is differentiable:
\[
\dot X_t = \frac{d X_t}{dt} = t^\beta \int_0^t s^\alpha (t-s)^\gamma \, dW_s .
\]

\end{frame}
\begin{frame}
\frametitle{The inversere presentation}
\framesubtitle{Case $\gamma\ge 0$}
\centerline{$t^{-\beta} \dot X_t = \displaystyle \int_0^t s^\alpha \, (t-s)^\gamma \, dW_s$.}
If $\gamma=0$, then \quad
$\displaystyle W_t = \int_0^t   s^{-\alpha}  \, d(s^{-\beta} \dot X_s)$.

If $\gamma\in\mathbb{N}$, then \quad
$\displaystyle W_t = \frac{1}{\gamma!}
 \int_0^t   s^{-\alpha}  \, d\!\left(
\frac{d^\gamma (s^{-\beta} \dot X_s)}
{ds^\gamma} \right)$.

If $\gamma>0$, then $t^{-\beta} \dot X_t$ is
a fractional antiderivative of $\Gamma(\gamma+1)\, \int_0^t s^\alpha dW_s$:
\qquad{$t^{-\beta} \dot X_t = \Gamma(\gamma+1)\, \mathcal{I}^{\gamma}_{0+}\!  \left(\int_0^t s^\alpha \, dW_s\right)$.}\\
Thus,
$\displaystyle W_t = \frac{1}{\Gamma(\gamma+1)}
 \int_0^t   s^{-\alpha}  \, d\!\left(
\mathcal{D}^{\gamma}_{0+} (s^{-\beta} \dot X_s)
\right)$.

For $\gamma = k - 1 + \{\gamma\}$,\quad $k\in\mathbb{N}$,\quad $0<\{\gamma\}<1$
\centerline{$\displaystyle
W_t = \frac{1}{\Gamma(\gamma{+}1)\, \Gamma(1{-}\{\gamma\})}
\int_0^t s^{-\alpha}
\, d\!\left(
\frac{d^k}{ds^k} \!\left(
\int_0^s u^{-\beta} (s-u)^{-\{\gamma\}} \, \dot X_u \, du \right)
\right).$}
\end{frame}


\end{document}


\documentclass{article}
\usepackage{amsmath,amssymb}
\DeclareMathOperator{\indicatorfun}{\mathbf{1}}
\DeclareMathOperator{\ME}{\mathsf{E}}
\DeclareMathOperator{\funceil}{ceil}
\def\dKdt(#1,#2){\frac{\partial K(#1,#2)}{\partial #1}}
\newcommand{\dotX}{\dot{X}}
\newcommand{\vS}{v}
\usepackage{amsthm}


\theoremstyle{plain}
\newtheorem{theorem}{Theorem}
\newtheorem{prop}{Proposition}
\newtheorem{lemma}{Lemma}
\newtheorem{corollary}{Corollary}
\theoremstyle{remark}
\newtheorem{remark}{Remark}
\theoremstyle{definition}
\newtheorem{definition}{Definition}
\title{Gaussian Volterra processes with power-type kernels. Part II}
\author{Yuliya Mishura and Sergiy Shklyar}

\begin{document}
\maketitle
\begin{abstract}
	
	In this paper, we study if the process is Lipschitz-continuous.
	We invert the Volterra representation
	and construct pathwise distant-future bounds for the process.
\end{abstract}

\section{Introduction and preliminaries}
Let $T>0$ and let $\{W_s,\;s\in[0,T]\}$ be a Wiener process.
We consider the Gaussian process with Volterra kernel (Gaussian Volterra process)
of the form
\begin{equation}\label{eq:VolterraPr}
X_t = \int_0^t K(t,s) \, dW_s,
\end{equation}
with
\begin{gather}
K(t,s) = s^\alpha \int_s^t u^\beta \, (u-s)^\gamma \, du 1_{s\le t},
\label{eq:KpowerV}
\end{gather}
so that \begin{gather}
X_t = \int_0^t s^\alpha \int_s^t u^\beta \, (u-s)^\gamma \, du \, dW_s.
\label{eq:XpowerV}
\end{gather}
According to \cite{Part 1}, condition $\int_0^tK^2(t,s)ds<\infty$ is   satisfied whenever
\begin{equation}
	\label{eq:XpowerVcond}
\alpha>-\frac12, \qquad
\gamma>-1, \quad \mbox{and} \quad
\alpha+\beta+\gamma > -\frac32.
\end{equation}
Additionally,   we proved in  \cite{Part 1} that under
condition \eqref{eq:XpowerVcond}
the process $X$
is H\"older continuous on the interval $[0,T]$
up to order $\min\bigl(1, \; \allowbreak
\gamma+\frac32, \; \allowbreak
\alpha+\beta+\gamma+\frac32\bigr)$.

Note that in the case where $\alpha=1/2-H, \beta=H-1/2, \gamma=H-3/2, H\in(1/2,1)$, process $X$ is a fractional Brownian motion.
For integration of a deterministic function
with respect to a fractional Brownian motion,
we refer to \cite{PapirasTaqqu2001}.
Various approaches to
stochastic integration with respect to a fractional
Brownian motion are developed in \cite{Biagini2008,Nualart2006}




\section{Integration with respect to the Volterra
process}
.

Let $\{X_t,\; t\mathbin{\in}[0,T]\}$
be a Volterra process defined in \eqref{eq:VolterraPr}
with kernel $K(t,s)$ of the form \eqref{eq:KpowerV},
and let $\phi(t)$ be a nonrandom function.
In \cite{Nualart2006},
for linear operator $K^*$ defined as
\[
	K^* \indicatorfun_{[0,t]} (s) = K(t,s),
\]
the integral with respect to the process $X$
is defined as
\[
	\int_0^T \phi(t) \, dX_t =
	\int_0^T K^*\phi (s) \, dW_s.
\]
Since in the case under consideration the kernel $K(t,s)$ is absolutely continuous
in $t$ and $K(s,s)=0,$ we can write
\[
	K(u,s) = \int_s^u \dKdt(t, s) \, dt,
\]
and so the obvious choice for operator $K^*$ is
\[
	(K^*\phi) (s) = \int_s^T \phi(t) \, \dKdt(t,s)\, dt.
\]
If it will be necessary to distinguish, operator $K^*$ will be marked as $K_{(\cdot,\cdot,\cdot)}^*$, by the values of the powers in the kernel $K$.
Now, 
\[
	\int_0^T \phi(t) \, dX_t
	= \int_0^T \int_s^T \phi(t) \, \dKdt(t,s) \, dt \, dW_s .
\]
So, the integral with respect
to the Volterra process \eqref{eq:XpowerV}
is formally defined as
\begin{equation}
	\label{eq:defLPI3Kernel}
\int_0^T \phi(t) \, dX_t = \int_0^T s^\alpha \int_s^T \phi(t) t^\beta (t-s)^\gamma \, dt \, dW_s.
\end{equation}

Now our goal is to define the class of functions $\phi$
for which the integral $\int_0^T \phi(t)\, dX_t$
is well-defined.
To that end, for $k\in  \mathbb{R}$ and $p\in[1,+\infty) $ we consider the Banach space
\[
	\mathcal{E}_{p,k} = \biggl\{\phi : [0,T]\mathbin{\to}
	\mathbf{R} \,:\,
	\mbox{$\phi$ is Borel measurable and~}
	\int_0^T |\phi(t)|^p\, t^{pk} \, dt < \infty\biggr\}
\]
of functions taken up to a. e.\@{} equivalence,
with the norm
\[
	\|\phi\|_{\mathcal{E}_{p,k}}
	= \left( \int_0^T |\phi(t)|^p \, t^{pk} \, dt
	\right)^{1/p} .
\]
Under additional condition $pk>-1$ the space
$\mathcal{E}_{p,k}$ contains indicators of intervals
$\mathbf{1}_{[0,t]}$, $0\mathbin{<}t\mathbin{\le}T$.
Regardless whether or not $pk>-1$,
the space
$\mathcal{E}_{p,k}$ contains indicators of intervals
$\mathbf{1}_{[t_0,t]}$, $0\mathbin{<}t_0\mathbin{<}t\mathbin{\le}T$,
and these indicators of intervals span
a dense subset in $\mathcal{E}_{p,k}$.

Now we need the next simple statement.

\begin{lemma}\label{lem:X123e3}
Let $a_i\in [0,1)$, $i=1,\ldots,3$,
be real numbers such that $a_1+a_2+a_3<2$.
Then there exist $\epsilon_i\in (0,1)$,
such that $a_i +  \epsilon_i\in(0,1)$,
$i=1,\ldots,3$,
and $\sum_{i=1}^3 (a_i+\epsilon_i) = 2$.
\end{lemma}
\begin{proof}
We can take
\[
\epsilon_i = \frac{(1-a_i) \, (2 - a_1 - a_2 - a_3)}
{3 - a_1 - a_2 - a_3}.
\]
Then  obviously
$\epsilon_i>0$,  $1-a_i-\epsilon_i>0$, and
 $\epsilon_1+\epsilon_2+\epsilon_3
= 2 - a_1 - a_2 -a_3$.
\end{proof}



\begin{prop}
\label{prop:PW-L-cont}
Let \eqref{eq:XpowerVcond} hold true, and let
  $p\mathbin{\ge}1$ and
\begin{equation} \label{forseven} p>\max\Bigl( \frac{2}{2\gamma+3},\:\allowbreak
     \frac{2}{3+2\alpha+2\beta+2\gamma}\Bigr).\end{equation}
Then the operator
\begin{equation}\label{operator}
(K^* \phi)(s) = s^\alpha \int_s^T \phi(t)\, t^\beta \, (t-s)^\gamma\, dt
\end{equation}
	
 
is a bounded linear operator $L^p[0,T]\to L^2[0,T]$.
\end{prop}

 
	 
	
\begin{proof}For $p=\infty$  Proposition~\ref{prop:PW-L-cont} follows
	from the proof of Theorem~1 \cite{Part 1}. So, let $p$ is finite, and 
 let's calculate the norms of functions on the interval
	$[0,T]$,  that is, $\|f\|_p$ is a short-hand notation for
	$\|f\|_{L^p[0,T]}$.
	With this notation,
	$$\|t^\alpha\|_p = (\alpha p + 1)^{-1/p}T^{\alpha + 1/p},$$
	as long as $p>0$ and $\alpha p > -1$.

	To prove the well-definedness and boundedness of the operator $K^*$,
	it suffices to show that
	\begin{equation}
		\label{eq:todoPW-L-cont}
		\int_0^T |\psi(s)|\, s^\alpha
		\int_s^T |\phi(t)|\, t^\beta (t-s)^\gamma
		\, dt \, ds \le
		c(p;\alpha,\beta,\gamma) \, \|\psi\|_2 \, \|\phi\|_p
	\end{equation}
	for all $\psi\in L^2[0,T]$ and
	for some   constant  $c(p;\alpha,\beta,\gamma)$ not depending  on
	$\psi$.
Consider six cases.

Case 1. Let $\gamma\ge 0$.
	Notice that under conditions of Proposition~\ref{prop:PW-L-cont},
	inequalities
	$\frac12+\alpha>0$ and $\frac1p - 1 - \beta - \gamma < \frac12+\alpha$
	hold true.
	Choose an arbitrary $\delta\ge 0$ such that
	$\frac1p - 1 - \beta - \gamma < \delta < \frac12+\alpha$.

	Obviously, in the  left-hand integral  in \eqref{eq:todoPW-L-cont} we have that
	$s\mathbin{\le}t$ and $t-s\mathbin{\le} t$.
	Therefore
	\begin{equation*}\begin{gathered}
	    \int_0^T |\psi(s)|\, s^\alpha
		\int_s^T |\phi(t)|\, t^\beta (t-s)^\gamma
		\, dt \, ds
		\le  \int_0^T |\psi(s)|\, s^{\alpha-\delta}
			\int_s^T |\phi(t)|\, t^{\delta+\beta+\gamma}
			\, dt \, ds
			\\ \le
			\int_0^T |\psi(s)|\, s^{\alpha-\delta} \, ds \,
			\int_0^T |\phi(t)|\, t^{\delta+\beta+\gamma} \, dt .
		 \end{gathered}\end{equation*}
	By H\"older's inequality,
	\begin{gather*}
		\int_0^T |\psi(s)|\, s^{\alpha-\delta} \, ds \le
		\|\psi\|_2 \, \|s^{\alpha-\delta}\|_2,
		 \int_0^T |\phi(t)|\, t^{\delta+\beta+\gamma} \, dt \le
		\|\phi\|_p \|t^{\delta+\beta+\gamma}\|_{p/(p-1)} .
	\end{gather*}
	Thus, \eqref{eq:todoPW-L-cont} holds true with
	\[
		c(p;\alpha,\beta,\gamma) =
		\|s^{\alpha-\delta}\|_2
		\|t^{\delta+\beta+\gamma}\|_{p/(p-1)},
	\]
	and the finiteness
	of the norms $\|s^{\alpha-\delta}\|_2$ and
	$\|t^{\delta+\beta+\gamma}\|_{p/(p-1)}$
	follows from the inequalities $p\ge 1$ and
	$\frac1p - 1 - \beta - \gamma < \delta < \frac12+\alpha$.

	 Case 2. Let $-1 < \gamma < 0$ and
$\frac{1}{p} < \gamma+1$.
	Let us find $\epsilon$ from the following equation:
	\begin{equation}
		\label{eq:eqfore}
		\frac{1}{p}
		+ \left( - \left( \alpha + {\textstyle\frac12} - \epsilon\right)^+
		  - \beta\right)^+
		- \gamma + 2\epsilon = 1 .
	\end{equation}
	It follows from assumption \eqref{forseven} that
	%$\frac{1}{p} < \gamma+1$ and
	$\frac{1}{p} < \frac32 + \alpha + \beta + \gamma$.
  Together with assumption $\frac{1}{p} <
\gamma + 1$, this imply that
	the left-hand side of \eqref{eq:eqfore}
	is less than 1 for $\epsilon=0$. In addition, the left-hand side of \eqref{eq:eqfore}  is obviously     greater than 1 for $\epsilon=\frac12$.
	Hence, there exists the unique $\epsilon\in(0,\frac12)$ that satisfies
	\eqref{eq:eqfore}; take it for the rest of the proof.

	As before,
	$s\mathbin{\le}t$, and thus
	\begin{equation}\label{eq:st1}
		s^{\left(\alpha+\frac12-\epsilon\right)^+} \le
		t^{\left(\alpha+\frac12-\epsilon\right)^+}.
	\end{equation}
	Since $\alpha - \bigl(\alpha+\frac12-\epsilon\bigr)^+ = \min\bigl(\alpha,\, \epsilon-\frac12\bigr)$,
	\begin{equation}\label{eq:st2}
		s^{\alpha - \left(\alpha+\frac12-\epsilon\right)^+}  t^\beta =
		s^{\min\left(\alpha,\, \epsilon-\frac12\right)} t^\beta.
	\end{equation}
	Multiplying \eqref{eq:st1} by \eqref{eq:st2}, we get
	\[
		s^\alpha t^\beta \le
		s^{\min\left(\alpha,\, \epsilon-\frac12\right)}
		t^{\left(\alpha+\frac12-\epsilon\right)^+ + \beta}.
	\]
	Hence,
	\begin{multline*}
		\int_0^T |\psi(s)|\, s^\alpha
		\int_s^T |\phi(t)|\, t^\beta (t-s)^\gamma
		\, dt \, ds
		\le \\ \le
		\int_0^T |\psi(s)|\,
		s^{\min\left(\alpha,\, \epsilon-\frac12\right)}
		\int_s^T |\phi(t)|\,
		t^{\left(\alpha+\frac12-\epsilon\right)^+ + \beta}
		(t-s)^\gamma
		\, dt \, ds .
	\end{multline*}
	Define $q$ and $r$ from the equations
	\begin{align*}
		\frac{1}{q} &=
		\left( - \left( \alpha + {\textstyle\frac12} - \epsilon\right)^+
		- \beta\right)^+ + \epsilon, &
		\frac{1}{r} &= \epsilon - \gamma,
	\end{align*}
respectively.
	Then $q>0$ and $r>0$ and, according to \eqref{eq:eqfore},
	$\frac{1}{p}+\frac{1}{q}+\frac{1}{r} = 1$.

	Denote
	\[
		h_s(t) = \begin{cases}
			0 & \mbox{if $t\in[0,s]$}, \\
			(t-s)^\gamma & \mbox{if $t\in(s,T]$}
		\end{cases}
	\]
	and apply H\"older inequality:
	\begin{gather*}
		\int_s^T |\phi(t)|\,
		t^{\left(\alpha+\frac12-\epsilon\right)^+ + \beta}
		(t-s)^\gamma
		\, dt=
		\int_0^T |\phi(t)|\,
		t^{\left(\alpha+\frac12-\epsilon\right)^+ + \beta}
		h_s(t)
		\, dt
		\le \\ \le
		 \|\phi\|_p
		\left\|t^{\left(\alpha+\frac12-\epsilon\right)^+ + \beta}\right\|_q
		\|h_s\|_r
		 \le
		\|\phi\|_p
		\left\|t^{\left(\alpha+\frac12-\epsilon\right)^+ + \beta}\right\|_q
		\|h_0\|_r .
	\end{gather*}
	Again, with H\"older inequality
	\begin{multline*}
		\int_0^T |\psi(s)|\, s^\alpha
		\int_s^T |\phi(t)|\, t^\beta (t-s)^\gamma
		\, dt \, ds
		\le \\
		\begin{aligned}
			&\le
			\int_0^T |\psi(s)|\,
			s^{\min\left(\alpha,\, \epsilon-\frac12\right)}
			\, dt \,
			\|\phi\|_p
			\left\|t^{\left(\alpha+\frac12-\epsilon\right)^+
				+ \beta}\right\|_q
			\|h_0\|_r
			\le \\ &\le
			\|\psi\|_2
			\left\|s^{\min\left(\alpha,\, \epsilon-\frac12\right)}\right\|_2
			\|\phi\|_p
			\left\|t^{\left(\alpha+\frac12-\epsilon\right)^+
				+ \beta}\right\|_q
			\|h_0\|_r .
		\end{aligned}
	\end{multline*}
	Inequality \eqref{eq:todoPW-L-cont} holds true with
	\begin{equation*}
		c(p; \alpha, \beta, \gamma) =
		\left\|s^{\min\left(\alpha,\, \epsilon-\frac12\right)}\right\|_2
		\left\|t^{\left(\alpha+\frac12-\epsilon\right)^+ + \beta}\right\|_q
		\|h_0\|_r .
	\end{equation*}

Case 3.
$-1<\gamma<0$, $\frac{1}{p}\ge \gamma+1$,
$\alpha\le0$ and $\beta\le 0$.
Due to assumption
\eqref{forseven},
$\frac{1}{p} < \frac32 - \alpha-\beta-\gamma$,
whence $\alpha+\beta>-\frac12$.
Apply Lemma~\ref{lem:X123e3}
for $X_1 = \frac12-\alpha-\beta$,
$X_2 = \frac{1}{p}$,
and
$X_3 =  -\gamma$, and find $\epsilon_i$
such that
\begin{gather*}
\frac12 - \alpha - \beta + \epsilon < 1,
\quad
\frac1p + \epsilon_2 < 1,
\quad
-\gamma + \epsilon_3 < 1, \\
\frac12 - \alpha - \beta + \epsilon +
\frac1p + \epsilon_2
-\gamma + \epsilon_3 = 2.
\end{gather*}
By H\"older inequality for non-conjugate
exponents
\[
  \|\psi(s) s^{\alpha+\beta}\|_
{1/(0.5+\epsilon_1-\alpha-\beta)}
\le
\|\psi\|_2
\|s^{\alpha+\beta}\|_{1/(\epsilon_1-\alpha-\beta)}.
\]
Also,
\[
\|\phi\|_{1/(p^{-1} + \epsilon_2)}
\le
\|\psi\|_p T^{\epsilon_2} .
\]
Since  $\beta\le 0$
and $s\le t$ on the integration domain
of \eqref{eq:todoPW-L-cont}, we have that
$t^\beta \le s^\beta$.

Then by Young's inequality
	\begin{align*}
		\int_0^T |\psi(s)|\, &s^\alpha
		\int_s^T |\phi(t)|\, t^\beta (t{-}s)^\gamma
		\, dt \, ds
 \le
		\int_0^T |\psi(s)|\, s^{\alpha+\beta}
		\int_s^T |\phi(t)|\, (t{-}s)^\gamma
		\, dt \, ds
\\ &\le
\|\psi(s) s^{\alpha+\beta}\|_{1/(0.5+\epsilon_1-\alpha-\beta)}
\|\phi\|_{1/(p^{-1}+\epsilon_2)}
\|u^\gamma\|_{1/(-\gamma+\epsilon_3)}
\\ &\le
\|\psi\|_2 \|s^{\alpha+\beta}\|_{1/(\epsilon_1-\alpha-\beta)} \,
\|\phi\|_p T^{\epsilon_2} \,
\|u^\gamma\|_{1/(-\gamma+\epsilon_3)},
\end{align*}
and \eqref{eq:todoPW-L-cont} holds true with
$c(p;\alpha,\beta,\gamma) =
\|s^{\alpha+\beta}\|_{1/(\epsilon_1-\alpha-\beta)} \,
T^{\epsilon_2} \,
\|u^\gamma\|_{1/(-\gamma+\epsilon_3)} .
$
	

Case 4. $-1<\gamma<0$, $\frac1p\ge\gamma+1$,
$\alpha\le 0$, $\beta\ge 0$, and
$\beta+\gamma<0$.
These imply $\beta<1$.
Hence
\begin{equation}
\label{neq:tst-s}
t^\beta < s^\beta + (t-s)^\beta
\quad \mbox{for all $s\in(0,t)$}
\end{equation}
due to concavity of $s^\beta + (t-s)^\beta$
and the fact that for $s=0$ and $s=t$
\eqref{neq:tst-s} becomes an equality.
Thus,
\begin{multline*}
		\int_0^T |\psi(s)|\, s^\alpha
		\int_s^T |\phi(t)|\, t^\beta (t{-}s)^\gamma
		\, dt \, ds
\\ \le
		\int_0^T\! |\psi(s)|\, s^{\alpha+\beta}
		\int_s^T\! |\phi(t)|\, (t{-}s)^\gamma
		\, dt \, ds
+
		\int_0^T\! |\psi(s)|\, s^\alpha
		\int_s^T\! |\phi(t)|\, (t{-}s)^{\beta+\gamma}
		\, dt \, ds .
\end{multline*}
Apply Lemma~\ref{lem:X123e3}
for $a_1 =\frac12 + (\alpha+\beta)^-$,
$a_2=\frac{1}{p}$ and $a_3 = -\gamma$.
Due to the same reason as in Case~3,
$\alpha+\beta>-\frac12$; hence,
$0<\frac12 + (\alpha+\beta)^-<1$.
Obviously, $0<\frac1p<1$ and $0<-\lambda<1$.
Due to \eqref{forseven},
\[
\frac1p < \frac32 + \alpha + \beta + \gamma
\quad \mbox{and} \quad
\frac1p < \frac32 + \gamma,
\]
whence
\[
\frac12 + (\alpha+\beta)^- + \frac1p - \gamma < 2.
\]
By Lemma~\ref{lem:X123e3}, there exist
$\epsilon_1$, $\epsilon_2$ and
$\epsilon_3$ such that
\begin{gather*}
\frac12 + (\alpha+\beta)^- + \epsilon_1
< 1, \quad
\frac{1}p + \epsilon_2 <1, \quad
-\lambda + \epsilon_3 < 1, \\
\frac12 + (\alpha+\beta)^- + \epsilon_1
+
\frac{1}p + \epsilon_2
-\lambda + \epsilon_3 = 2.
\end{gather*}
By Young's inequality and H\"older inequality
for non-conjugate exponents,
	\begin{align*}
		\int_0^T |\psi(s)|\, s^{\alpha+\beta}
		&\int_s^T |\phi(t)|\, (t{-}s)^\gamma
		\, dt \, ds
\\ &\le
\|\psi(s) s^{\alpha+\beta}\|_{1/(0.5+(\alpha+\beta)^- + \epsilon_1)}
\|\phi\|_{1/(p^{-1}+\epsilon_2)}
\|u^\gamma\|_{1/(-\gamma+\epsilon_3)}
\\ &\le
\|\psi\|_2 \|s^{\alpha+\beta}\|_{1/((\alpha+\beta)^- + \epsilon_1)} \,
\|\phi\|_p T^{\epsilon_2} \,
\|u^\gamma\|_{1/(-\gamma+\epsilon_3)}.
\end{align*}

Apply Lemma~\ref{lem:X123e3} again,
this time for $a_1=\frac12 - \alpha$,
$a_2 = \frac{1}{p}$ and $a_3 = -\beta-\gamma$.
Here conditions of Lemma~\ref{lem:X123e3}
are easier to check.
By Lemma~\ref{lem:X123e3}, there exist
$\delta_1$, $\delta_2$ and $\delta_3$
such that
\begin{gather*}
\frac12 - \alpha + \delta_1  < 1, \quad
\frac{1}{p} + \delta_2 < 1, \quad
-\beta-\gamma+\delta_3<1, \\
\frac12 - \alpha + \delta_1 +
\frac{1}{p} + \delta_2 < 1
-\beta-\gamma+\delta_3 = 2.
\end{gather*}
By Young's inequality
 and H\"older inequality
for non-conjugate exponents,
	\begin{align*}
		\int_0^T |\psi(s)|\, s^\alpha
		&\int_s^T |\phi(t)|\, (t{-}s)^{\beta+\gamma}
		\, dt \, ds
\\ &\le
\|\psi(s) s^\alpha\|_{1/(0.5-\alpha + \delta_1)}
\|\phi\|_{1/(p^{-1}+\delta_2)}
\|u^\gamma\|_{1/(-\gamma-\beta+\delta_3)}
\\ &\le
\|\psi\|_2 \|s^\alpha\|_{1/(\delta_1-\alpha)} \,
\|\phi\|_p T^{\delta_2} \,
\|u^\gamma\|_{1/(\delta_3-\beta-\gamma)}.
\end{align*}
Thus, inequality \eqref{eq:todoPW-L-cont}
holds true with
\begin{align*}
c(p;\alpha,\beta,\gamma)
&=
\|s^{\alpha+\beta}\|_{1/((\alpha+\beta)^- + \epsilon_1)} \,
 T^{\epsilon_2} \,
\|u^\gamma\|_{1/(-\gamma+\epsilon_3)}
\\ &+
\|s^\alpha\|_{1/(\delta_1-\alpha)} \,
 T^{\delta_2} \,
\|u^\gamma\|_{1/(\delta_3-\beta-\gamma)}.
\end{align*}

Case 5.
$-1<\gamma<0$, $\frac1p\ge\gamma+1$,
$\alpha\le 0$, $\beta\ge 0$, and
$\beta+\gamma\ge0$.
Similarly to \eqref{neq:tst-s},
\begin{equation*}
t^{-\gamma} < s^{-\gamma} + (t-s)^{-\gamma}
\quad \mbox{for all $s\in(0,t)$}.
\end{equation*}
Thus,
\begin{multline*}
		\int_0^T |\psi(s)|\, s^\alpha
		\int_s^T |\phi(t)|\, t^\beta (t{-}s)^\gamma
		\, dt \, ds
\\ \le
		\int_0^T\! |\psi(s)|\, s^{\alpha-\gamma}
		\int_s^T\! |\phi(t)|\, t^{\beta+\gamma} (t{-}s)^\gamma
		\, dt \, ds
+
		\int_0^T\! |\psi(s)|\, s^\alpha\,
		\int_s^T\! |\phi(t)|\, t^{\beta+\gamma}
		\, dt \, ds
\\ \le
		\int_0^T\! |\psi(s)|\, s^{\alpha-\gamma}
		\int_s^T\! |\phi(t)|\, t^{\beta+\gamma} (t{-}s)^\gamma
		\, dt \, ds
+
		\int_0^T\! |\psi(s)|\, s^\alpha\,  ds\,
		\int_0^T\! |\phi(t)|\, t^{\beta+\gamma}
		\, dt.
\end{multline*}
Apply Lemma~\ref{lem:X123e3}
for $a_1 = \frac12 + (\alpha-\gamma)^-$,
$a_2 = \frac{1}{p}$ and
$a_3 = -\gamma$.
From \eqref{eq:todoPW-L-cont} it follows
that $\frac12 + \frac{1}{p}  - \gamma < 2$.
Since $\alpha>-\frac12$ and $p\ge 1$,
$\frac12 - \alpha + \frac{1}{p}< 2$.
Hence,
\[
\frac12 + (\alpha-\gamma)^- + \frac{1}{p}
- \gamma
= \frac12 - \min(\alpha,\gamma) + \frac{1}{p} < 2.
\]
By Lemma~\ref{lem:X123e3} there exist
$\epsilon_1>0$, $\epsilon_2>0$ and
$\epsilon_3>0$ such that
\begin{gather*}
\frac12 + (\alpha-\gamma)^- + \epsilon_1 < 1,
\quad
\frac{1}{p} + \epsilon_2 < 1,
\quad
\epsilon_3 - \gamma < 1, \\
\frac12 + (\alpha-\gamma)^- + \epsilon_1 +
\frac{1}{p} + \epsilon_2 +
\epsilon_3 - \gamma = 2.
\end{gather*}
By Young's inequality and
H\"older inequality for non-conjugate exponents,
\begin{align*}
\int_0^T\! &|\psi(s)|\, s^{\alpha-\gamma}
		\int_s^T\! |\phi(t)|\, t^{\beta+\gamma} (t{-}s)^\gamma
		\, dt \, ds
\\ &\le
\|\psi(s) s^{\alpha-\gamma}\|
_{1/(0.5+(\alpha-\gamma)^- + \epsilon_1)}
\|\phi(t) t^{\beta+\gamma}\|
_{1/(p^{-1}+\epsilon_2)}
\|u^\gamma\|
_{1/(\epsilon_3-\gamma)}
\\ &\le
\|\psi\|_2
\|s^{\alpha-\gamma}\|
_{1/((\alpha-\gamma)^- + \epsilon_1)} \,
\|\phi\|_p
\|t^{\beta+\gamma}\|
_{1/\epsilon_2} \,
\|u^\gamma\|_{1/(\epsilon_3-\gamma)} .
\end{align*}
By H\"older inequality
\begin{gather*}
\int_0^T |\psi(s)| s^\alpha \, ds \le
\|\psi\|_2  \|s^\alpha\|_2,
\qquad
\int_0^T |\phi(t)| t^{\beta+\gamma} \, dt \le
\|\phi\|_p  \|t^{\beta+\gamma}\|_{p/(p-1)}.
\end{gather*}
Thus, inequality \eqref{eq:todoPW-L-cont}
holds true with
\begin{align*}
c(p;\alpha,\beta,\gamma)
&=
\|s^{\alpha-\gamma}\|_{1/((\alpha-\gamma)^- + \epsilon_1)} \,
\|t^{\beta+\gamma}\|_{1/\epsilon_2} \,
\|u^\gamma\|_{1/(\epsilon_3-\gamma)}
\\ &+
\|s^\alpha\|_2 \,
\|t^{\beta+\gamma}\|_{p/(p-1)}.
\end{align*}

Case 6.
$\alpha > 0$.
We have already proved Proposition~\ref{prop:PW-L-cont}
for $\alpha\le 0$.  We are going
to use it for $\alpha=0$.

On the integration domain of
\eqref{eq:todoPW-L-cont} $s\le t$,
and so $s^\alpha \le t^\alpha$.
We have
	\begin{align*}
		\int_0^T |\psi(s)|\, s^\alpha
		\int_s^T |\phi(t)|\, t^\beta (t{-}s)^\gamma
		\, dt \, ds &\le
		\int_0^T |\psi(s)|\,
		\int_s^T |\phi(t)|\, t^{\alpha+\beta} (t{-}s)^\gamma
		\, dt \, ds
\\ & \le
		c(p; 0, \alpha{+}\beta, \gamma) \, \|\psi\|_2 \, \|\phi\|_p.
	\end{align*}
The inequality \eqref{eq:todoPW-L-cont}
holds true for
$c(p; \alpha, \beta,\gamma) = c(p; 0, \alpha{+}\beta, \gamma)$.
\end{proof}

 

\begin{corollary}
	\label{corollary:ELbound}
	Let
	\begin{gather*}
		\alpha>-\frac12,
		\qquad
		p\ge 1,
		\qquad
		\frac{1}{p} < \frac32+\gamma, \qquad
		\frac{1}{p} + k < \frac32 + \alpha+\beta+\gamma .
	\end{gather*}
	Then the operator $K^*$ defined in \eqref{operator}
	is a bounded linear operator $\mathcal{E}_{p,k}\to  L^2[0,T]$.
\end{corollary}
\begin{proof}
Recall that
the operator $K^*$ for specific $\alpha$, $\beta$ and
$\gamma$ is denoted by  $K^*_{(\alpha,\beta,\gamma)}$.
From the definition, it follows that
\begin{gather*}
(K^*_{(\alpha,\beta,\gamma)}
(\phi(t) t^k))  (s) =
s^\alpha \int_s^T \! \phi(t) \, t^{k+\beta} \, (t-s)^\gamma\, dt
=(K^*_{(\alpha,\beta+k,\gamma)} \phi) (s).
\end{gather*}
Then
\begin{align}
\|K^*_{(\alpha,\beta,\gamma)} \phi\|_2
&=
\|K^*_{(\alpha,\beta-k,\gamma)} (\phi(t) t^k)\|_2
\nonumber \\ &\le
c(p; \alpha, \beta{-}k, \gamma) \, \|\phi(t) t^k\|_p
=
c(p; \alpha, \beta{-}k, \gamma) \, \|\phi\|_{\mathcal{E}_{p,k}}
\label{eq:proofcor1}
\end{align}
for any function $\phi$ such that
$K^*_{(\alpha,\beta-k,\gamma)} (\phi(t) t^k)$ is well-defined.
Under conditions of Corollary~\ref{corollary:ELbound},
the operator
$K^*_{(\alpha,\beta-k,\gamma)} : L^p[0,T] \to L^2[0,T]$
is bounded and $c(p; \alpha, \beta{-}k, \gamma)$ can be chosen as a finite number.
Thus,
the boundedness of the operator
$K^*_{(\alpha,\beta,\gamma)} : \mathcal{E}_{p,k} \to L^2[0,T]$
follows from \eqref{eq:proofcor1}.
\end{proof}

In the following definition conditions \eqref{eq:XpowerVcond}
	are satisfied, $X$ is a process
	defined by \eqref{eq:XpowerV}, and
\[
	1 \le  p < \infty, \qquad
	p k > -1, \qquad
	\frac{1}{p} < \frac32 +\gamma, \qquad
	\frac{1}{p} + k < \frac32 + \alpha+\beta+\gamma .
\]
\begin{definition}\label{def-PLI-3p}
	The integral of a non-random function
	$\phi\in\mathcal{E}_{p,k}$
	with respect to the Gaussian process $X$
	is defined as
	\begin{equation}\label{eq:def-PLI-3p}
		\int_0^T \phi(t) \, dX_t
		= \int_0^T K^* \phi(s) \, dW_s
		= \int_0^T s^\alpha
		\int_s^T \phi(t) t^\beta (t-s)^\gamma \, dt
		\, dW_s.
	\end{equation}
\end{definition}
Particularly, Definition \ref{def-PLI-3p} can be applied for
\[
	\alpha>-\frac12,
	\quad
	1 \le p < \infty,
	\quad
	\frac{1}{p} < \frac32+\gamma, \quad
	\frac{1}{p} < \frac32 + \alpha+\beta+\gamma,
	\quad \mbox{and} \quad
	\phi \in L^p[0,T].
\]

\begin{remark}\label{rem:just-def-PLI-3p}
 Under conditions of Corollary~\ref{corollary:ELbound},
\eqref{eq:def-PLI-3p} provides linear continuous
mapping $\mathcal{E}_{p,k} \to L^2(\Omega,\mathcal{F},\mathsf{P})$,
where $L^2(\Omega,\mathcal{F},\mathcal{P})$
is a space of square-integrable random variables.
Furthermore, if $p k\mathbin{>} {-}1$,
then \eqref{eq:def-PLI-3p}
gives $\int_0^T \indicatorfun_{[a,b]} dX_t = X_b - X_a$,
which is very natural.
If $1\mathbin{\le}p\mathbin{<}\infty$,
then the indicators of intervals span a dense subset
in $\mathcal{E}_{p,k}$.
Thus, in Definition~\ref{def-PLI-3p},
we construct a linear and continuous extension
of integrals of indicators of intervals.
\end{remark}

\section{Differentiability of Volterra process}
Consider the stochastic process
\begin{equation}\label{eq:dotXdef}
\dotX _t = t^\beta \int_0^t s^\alpha (t-s)^\gamma \, dW_s .
\end{equation}
It is well-defined if
\[
\alpha>-\frac{1}{2} \quad \mbox{and}  \quad
\gamma>-\frac{1}{2},
\]
and
is self-similar with exponent $\alpha+\beta+\gamma+\frac12$.
Let  $\lceil x \rceil$ stand for 
  the least integer that is greater   or equal to $x$.

\begin{prop}
If $\alpha>-\frac{1}{2}$, $\beta\in\mathbb{R}$
and $\gamma\in\bigl(-\frac12, \frac12\bigr]$,
then the process $\dotX$ has a modification
which is H\"older-continuous
up to order $\gamma+\frac12$ on any interval $[t_0, T]$,
$0<t_0<T$.


If $\alpha>-\frac{1}{2}$, $\beta\in\mathbb{R}$
and $\gamma>\frac12$, then the process
$\dotX$ has a modification
which is continuously differentiable on $(0,+\infty)$,
and, as the result, is Lipschitz continuous
on any interval $[t_0,T]$.
Moreover, the process $\dotX$ is $\lceil \gamma-\frac12 \rceil$
times continuously differentiable. 
\end{prop}

\begin{proof}
The process $M_t = \int_0^t s^\gamma \, dW_s$
has a modification that satisfies H\"older
condition up to order $\frac12$ on any interval $[t_0, T]$, $t_0\mathbin{\in}(0, T)$.

First, consider the case $\gamma\in\bigl(-\frac12,\frac12\bigr]$.
Let $\lambda \in \bigl(0 \mathbin{\vee} \gamma,\,\allowbreak \gamma+\frac12
\bigr)$.
The process $M$ satisfies H\"older condition of
order $\lambda-\gamma \in \bigl(0, \frac12)$
on the interval $[t_0, T]$.
Then $\int_0^t s^\alpha (t-s)^\gamma \, dW_s = \int_0^t
(t-s)^\gamma \, dM_s$
satisfies H\"older condition of order $\lambda$
due to \cite[Lemma~2.1]{Norros1999}.
Thus, the process $\dotX$ also satisfies
H\"older condition of order $\lambda$ on $[t_0,T]$.

Second, consider the case $\gamma > \frac12$.
We can choose a positive integer $k\in\mathbb{N}$ and a real number
$\lambda_0 \in \bigl(0,\frac12\bigr)$
such that $0<\gamma - k + \lambda_0 < 1$.
	(We can choose $k = \lceil \gamma-\frac12\rceil$,
	and $\lceil \gamma-\frac12\rceil \vee 0 < \lambda_0 < \frac12$,
	as $\lceil \gamma-\frac12 \rceil < \gamma+\frac12$.)
	%Here $\lceil x \rceil = \funceil(x)$ is a rounding upwards
	%function.
The process $M$ is H\"older-continuous
of order $\lambda_0$ on any interval $[t_0, T]$.
Then the process
$\int_0^t s^\alpha (t-s)^{\gamma-k} \, dW_s = \int_0^t
(t-s)^{\gamma-k} \, dM_s$
is H\"older continuous of order
$\gamma - k + \lambda_0$ on any interval $[t_0, T]$, and thus
continuous on $(0,\, {+}\infty)$.
The process
\[
\int_0^t s^\alpha (t-s)^\gamma \, dW_s
= \frac{\Gamma(\gamma{+}1)\, }  {(k{-}1)! \,  \Gamma(\gamma{-}k{+}1)}
\int_0^t (t-u)^{k-1} \int_0^u s^\alpha (u-s)^{\gamma-k} \, dW_s  \, du
\]
is the $k$th antiderivative of the process
$\frac{\Gamma(\gamma{+}1)\, }  { \Gamma(\gamma-k+1)}
\int_0^t s^\alpha (t-s)^{\gamma-k} \, dW_s$,
and thus is $k$ times continuously differentiable on
$(0,\, {+}\infty)$.
As the result, the process $\dotX$
is $k$ times continuously differentiable on
$(0,\, {+}\infty)$.
It satisfies the Lipschitz condition on
any interval $[t_0, T]$.
\end{proof}
The next general  result is in fact the part of Lemma 2 from \cite{Part 1}, and it will be applied in the proof of Proposition \ref{prop3}.

\begin{lemma}\label{lem:Lemma-upperhelix}
Let the process $Y$ be self-similar   with exponent $H>0$ and be 
mean-square continuous on $[0,T]$. Additionally, let 
the process $Y$ satisfy the inequality
\[
\ME(Y_T - Y_t)^2 \le C\, T^{2H-2\lambda} (T - t)^{2\lambda} , \qquad
t_0\mathbin{<}t\mathbin{<}T 
\]
for some
$C>0$, $\lambda>0$ and for any $0<t_0<T$.
Then there exists $c>0$ such that
\[
\ME(Y_t - Y_s)^2 \le c\, (t-s)^{2 (\lambda \wedge H)} , \qquad
0\mathbin{\le}s\mathbin{<}t\mathbin{\le}T.
\]
\end{lemma}

Now let us return to our process $X$.

\begin{prop}\label{prop3}
If $\alpha>-\frac12$ and $\gamma>-\frac12$,
then the process $\dotX$ is mean-square continuous
and has continuous modification on $(0,T]$.
If, in addition, $\alpha+\beta+\gamma>-\frac12$,
then the process $\dotX$ is mean-square continuous
and has continuous modification on $[0,T]$.
\end{prop}

\begin{proof}
Path-continuity of the process $\dotX$ on $(0,T)$
follows from local H\"older or Lipschitz condition.
The process $\dotX$ satisfies H\"older condition
up to order $(\gamma+\frac12)\vee 1$ on any finite
interval  $[t_0,T]$, $0<t_0<T$.
According to \cite[Theorem 1]{ASVY2014},
for any $\lambda_0$,
$0<\lambda_0 <(\gamma+\frac12)\vee 1$
the incremental variance satisfies
\[
\ME(\dotX_t - \dotX_s)^2
\le C(t_0, T, \lambda_0) \, |t-s|^{2\lambda_0},
\]
whence the mean-square continuity on $(0,T]$ follows.
Furthermore,
\[
\ME \dotX_t^2 =  \ME(\dotX_t - \dotX_0)^2 =
t^{2\alpha+2\beta+2\gamma+1} \, \mathrm{B}(2\alpha{+}1, \, 2\gamma{+}1),
\]
Under additional condition $\alpha+\beta+\gamma>-\frac12$
the process $\dotX$ is mean-square continuous at point $0$.
Due to Lemma~\ref{lem:Lemma-upperhelix},
the process $\dotX$ satisfies the inequality
\[
\ME(\dotX_t - \dotX_s)^2 \le
C_1 \, (t-s)^{(2\lambda_0) \wedge (2\alpha+2\beta+2\gamma+1)},
\qquad 0\mathbin{\le}s\mathbin{<}t\mathbin{\le}T .
\]
Hence, the process $\dotX$ has a modification that is
H\"older-continuous up to order $\lambda_0 \wedge \bigl(\alpha+\beta+\gamma+\frac12
\bigr)$ on the interval $[0,T]$ and thus is continuous at point $0$.
\end{proof}


The continuity of $\dotX$ allows to change the order of integration.
For the process $X$ defined in \eqref{eq:XpowerV},
\begin{align*}
X_{t_2}-X_{t_1}
&=
\int_0^{t_2} s^\alpha \int_{s\vee t_1}^{t_2} u^\beta \, (u-s)^\gamma \, du \, dW_s
\\ &=
\int_{t_1}^{t_2} u^\beta \int_0^u s^\alpha \, (u-s)^\gamma \, dW_s \, du
= \int_{t_1}^{t_2} \dotX_u \, du .
\end{align*}

\begin{corollary}\label{corollary:dXdt}
If $\alpha>-\frac12$, $\gamma>-\frac12$ and $\alpha+\beta+\gamma>-\frac32$,
then some modifications of the processes $X$ and $\dotX$ defined in
\eqref{eq:XpowerV} and \eqref{eq:dotXdef} satisfy
the relation
\[
	X_t = X_{t_0} + \int_{t_0}^t \dotX_s \, ds.
\]
The process $X_t$ is $\lceil\gamma+\frac12\rceil$ times
	continuously differentiable on $(0,{+}\infty)$,
	and
	\begin{equation}\label{eq:dXtdXt}
		\frac{d X_t}{dt} = \dotX_t,\quad t>0.
	\end{equation}

If, in addition, $\alpha+\beta+\gamma>-\frac12$,  then
the processes $X$ and $\dotX$ satisfy the relation
\begin{equation}\label{eq:Xtdif}
X_t = \int_0^t \dotX_s \, ds.
\end{equation}
Then the process $X_t$ is continuously differentiable on $[0,{+}\infty)$.

Representation \eqref{eq:Xtdif} is also valid for
$\alpha+\beta+\gamma \in \bigl(-\frac32, -\frac12\bigr]$.
However, in this case, the integrand $\dotX$ may be unbounded
in the neighborhood of $0$.  Thus,
the integral here should be understood in improper sense:
	\[
		X_t = \lim_{t_0\to 0+} \int_{t_0}^t \dotX_s \, ds.
	\]
\end{corollary}




\section{Inverse representation}
In Volterra representation \eqref{eq:VolterraPr}
(or, specifically, \eqref{eq:XpowerV})
the process $X$ is represented as the integral
with respect to the Wiener process $W$.
We construct the representation
of the Wiener process $W$ in \eqref{eq:XpowerV}
using the process $X$.
However, the form of the representation
depends on whether $\gamma\in(-1,0)$
or $\gamma\ge 0$, and on whether or not $\gamma$ is integer.


\subsection[Case $\gamma<0$]{Case \boldmath$\gamma<0$}
\subsubsection*{Reduction to the integral equation}
We are going to find the inverse representation to \eqref{eq:VolterraPr}:
\begin{equation}\label{invrep}
W_\vS = \int_0^\vS L(\vS,t) \, dX_t \, .
\end{equation}
Assume that integral with respect to $X$ is performed according
to \eqref{eq:defLPI3Kernel}. %\eqref{eq:intQ}.
Then the left-hand side and the right-hand side of \eqref{invrep} admit the representations
\begin{align*}
W_\vS &= \int_0^\vS dW_\vS, \\
\int_0^\vS L(\vS,t) \, dX_t &= \int_0^\vS \int_s^\vS L(\vS,t)\, \dKdt(t,s) \, dt\, dW_s.
\end{align*}
Thus,
the sufficient and necessary condition for \eqref{invrep} is
\begin{equation}\label{eq:inIEQ-V}
\int_s^\vS L(\vS,t) \dKdt(t,s) \, dt = 1
\end{equation}
for all $\vS\in(0,T]$ and for almost all $s\in(0,\vS)$.
For the kernel defined %in \eqref{eq:KVol} and
\eqref{eq:KpowerV},
the integral equation \eqref{eq:inIEQ-V} turns into
%\begin{equation}\label{eqLinIEQ-abc}
%\int_s^\vS L(\vS,t) a(s) b(t) c(t-s) \, dt = 1
%\end{equation}
%and
\begin{equation}\label{eqLinIEQ-power}
\int_s^\vS L(\vS,t)\, s^\alpha\, t^\beta\, (t-s)^\gamma \, dt = 1.
\end{equation}
\subsubsection*{The explicit solution}
[Perhaps, this paragraph should be moved to the introduction.]
The solution to equation
\[
\int_s^\vS L(\vS,t) a(s) b(t) c(t-s) \, dt = 1
\]
is found in
Mishura \textit{et al}.\@{} \cite[Proposition 3]{arXiv03405}.
In that solution, they use the ``Sonine pair'' with the function $c(s)$, which is
a function $p(s)$ that satisfies the integral equation
\[
\int_0^t p(s) \, c(t-s) \, ds = 1, \qquad t\in(0,T].
\]
The ``Sonine pair'' with the power function $s^\gamma$ exists for all $\gamma\in(-1,0)$;
it is equal to $s^{-\gamma-1} / \, \mathrm{B}(-\gamma, \gamma{+}1)$.


Now we construct a solution to equation \eqref{eqLinIEQ-power}
within a class of functions $L(\vS,t)$, $0<t<\vS<T$,
that satisfy the following absolute integrability condition
\begin{equation}\label{eq:abocon}
\int_s^\vS |L(\vS,t)| \, dt < \infty, \qquad
0\mathbin{<}s\mathbin{<}\vS\mathbin{\le}T.
\end{equation}



% is unique up to equal-almost-everywhere equivalence in $t$.
% In other words, if two functions $L_1$ and $L_2$
% satisfy both \eqref{eqLinIEQ-power} and \eqref{eq:abocon},
% then for all $\vS\in(0,T]$
% \[
% \lambda_1\{t\mathbin{\in}(0,\vS) : L_1(\vS,t) \mathbin{=} L_2(\vS,t)\} = \vS
% \quad \mbox{and} \quad
% \lambda_1\{t\mathbin{\in}(0,\vS) : L_1(\vS,t) \mathbin{\neq} L_2(\vS,t)\} = 0,
% \]
% where $\lambda_1$ is the Lebesgue measure.
\begin{theorem}\label{thm:Lker-suf}
Let $\alpha\mathbin{>}{-}\frac12$,\spacefactor3000{}
$\beta\mathbin{\in}\mathbb{R}$, and
${-}1\mathbin{<}\gamma\mathbin{<}0$.
Then
the solution to the integral equation~\eqref{eqLinIEQ-power}
for all $s$ and $\vS$ such that $0<s<\vS\le T$
is
\begin{align*}
L(\vS,t) &=
- \frac{t^{-\beta}}{\mathrm{B}(\gamma{+}1, {-}\gamma)} \,
\frac{\partial}{\partial t}\!
\left( \int_t^\vS s^{-\alpha} (s-t)^{-\gamma-1} \, ds \right)
\\ &=
 \frac{1}{\mathrm{B}(-\gamma, \gamma{+}1)\,t^\beta}
\left(\vS^{-\alpha} (\vS-t)^{-\gamma-1}
+ \alpha \int_t^\vS u^{-\alpha-1} (u-t)^{-\gamma-1} \,du \right).
\end{align*}

The solution is unique up to equality for all $\vS\in(0,T]$
and for almost all $t\in(0,\vS)$.
\end{theorem}



\begin{proof}
\textit{Necessity.}
	Let $L(t,s)$ satisfies both \eqref{eqLinIEQ-power}
	and \eqref{eq:abocon}.
	Let us find the expression for $L(t,s)$.

Let $0<r<\vS\le T$.
Calculate the double integral
\begin{equation}\label{eq:iiexpress}
\iint\limits_{r<s<t<\vS} L(\vS,t)
t^\beta\, (t-s)^\gamma (s-r)^{-\gamma -1} \, dt \, ds.
\end{equation}
First, proof the existence.
\begin{multline*}
\iint\limits_{r<s<t<\vS} |L(\vS,t)
t^\beta\, (t-s)^\gamma (s-r)^{-\gamma -1}| \, dt \, ds
= \\
\begin{aligned}
&=
\int_r^\vS t^\beta \, |L(\vS,t)| \int_r^t (t-s)^\gamma (s-r)^{-\gamma -1} \, ds \, dt
= \\ &=
\mathrm{B}(\gamma{+}1, {-}\gamma)
\int_r^\vS t^\beta \, |L(\vS,t)| \, dt
\le \\ & \le
\mathrm{B}(\gamma{+}1, {-}\gamma)\,
\max(r^\beta,\, \vS^\beta)
\int_r^\vS |L(\vS,t)| \, dt < \infty .
\end{aligned}
\end{multline*}
Thus, the integral in \eqref{eq:iiexpress} exists absolutely.
Calculate it in two ways.
On the one hand,
\begin{multline*}
\iint\limits_{r<s<t<\vS} L(\vS,t)
t^\beta\, (t-s)^\gamma (s-r)^{-\gamma -1} \, dt \, ds
= \\
\begin{aligned}
&=
\int_r^\vS t^\beta \, L(\vS,t) \int_r^t (t-s)^\gamma (s-r)^{-\gamma -1} \, ds \, dt
= \\ &=
\mathrm{B}(\gamma{+}1, {-}\gamma)
\int_r^\vS t^\beta \, L(\vS,t) \, dt .
\end{aligned}
\end{multline*}
On the other hand, with \eqref{eqLinIEQ-power},
\begin{multline*}
\iint\limits_{r<s<t<\vS} L(\vS,t)
t^\beta\, (t-s)^\gamma (s-r)^{-\gamma -1} \, dt \, ds
= \\
\begin{aligned}
&=
\int_r^\vS s^{-\alpha} \int_s^\vS L(\vS,t) \, s^\alpha t^\beta (t-s)^\gamma \, dt
\, (s-r)^{-\gamma-1} \, ds
= \\ &=
\int_r^\vS s^{-\alpha} (s-r)^{-\gamma-1} \, ds.
\end{aligned}
\end{multline*}
Thus, $L(\vS,t)$ satisfies the equation
\[
\mathrm{B}(\gamma{+}1, {-}\gamma)
\int_r^\vS t^\beta \, L(\vS,t) \, dt
=
\int_r^\vS s^{-\alpha} (s-r)^{-\gamma-1} \, ds.
\]
Hence $L(\vS,t)$ can be expressed explicitly:
\begin{equation}\label{eq:Lviadif}
L(\vS,t) = - \frac{t^{-\beta}}{\mathrm{B}(\gamma{+}1, {-}\gamma)} \,
\frac{\partial}{\partial t}\!
\left( \int_t^\vS s^{-\alpha} (s-t)^{-\gamma-1} \, ds \right)
\end{equation}
for almost all $t\in(0,\vS)$.

\textit{Sufficiency.}
First, consider the generic case $\alpha\neq 0$, that is either $\alpha\in\bigl(\frac12,0\bigr)$
or $\alpha\mathbin{>}0$. Let $0<s<\vS\le T$.
Then
\begin{gather*}
\int_s^\vS \frac{\vS^{-\alpha}}{(\vS-t)^{\gamma+1}}\,
(t-s)^\gamma \, dt
= \frac{B(-\gamma, \gamma{+}1)}{\vS^\alpha} ,
\displaybreak[1]\\
\begin{aligned}
\int_s^\vS \int_t^\vS \frac{u^{-\alpha-1} \, du}{(u-t)^{\gamma+1}}
\, (t-s)^\gamma \,dt
&=
\int_s^\vS u^{-\alpha-1} \int_s^u \frac{(t-s)^\gamma \, dt}{(u-t)^{\gamma+1}}
\, du
= \\ &=
\int_s^\vS u^{-\alpha-1} \mathrm{B}(\gamma{+}1,-\gamma)
\, du
= \\ &=
\frac{1}{\alpha} \, B(-\gamma, \gamma{+}1) \, (s^{-\alpha} - \vS^{-\alpha}) .
\end{aligned}
\end{gather*}
Hence
\begin{multline*}
\int_s^\vS L(\vS,t) \, s^\alpha t^\beta (t-s)^\gamma \, dt
= \\
\begin{aligned}
&=
\frac{s^\alpha}{B(-\gamma, \gamma{+}1)}
\int_s^\vS
\left(
\frac{\vS^{-\alpha}}{(\vS-t)^{\gamma+1}} +
\alpha \int_t^\vS \frac{u^{-\alpha-1} \, du}{(u-t)^{\gamma+1}} \right)
(t-s)^\gamma \, dt
= \\ &=
\frac{s^\alpha}{B(-\gamma, \gamma{+}1)}
\left(\frac{B(-\gamma, \gamma{+}1)}{\vS^\alpha} + B(-\gamma, \gamma{+}1) \, (s^{-\alpha} - \vS^{-\alpha}) \right)
= 1 .
\end{aligned}
\end{multline*}
For $\alpha\neq 0$, the theorem is proved.

Now, consider the case $\alpha=0$. Then
\[
L(\vS,t) = \frac{1}{B(-\gamma, \gamma{+}1) \, t^\beta (\vS-t)^{\gamma+1}} .
\]
Then, for all $s$ and $\vS$ such that $0<s<\vS\le T$
\[
\int_s^\vS
L(\vS,t) \, t^\beta (t-s)^\gamma \, dt
=
\frac{1}{B(-\gamma, \gamma{+}1)} \int_s^\vS
\frac{(t-s)^\gamma \, dt}{(\vS-t)^{\gamma+1}} = 1 .
\]

It remains to verify that the function $L(\vS,t)$
satisfies condition \eqref{eq:abocon}.
The factor $t^{-\beta} / \mathrm{B}(-\gamma, \gamma{+}1)$
is bounded for $t\in(s,\vS)$.
As to the other factor,
\begin{multline*}
	\int_s^\vS \left|
	\vS^{-\alpha} (\vS-t)^{-\gamma-1}
	+ \alpha \int_t^\vS u^{-\alpha-1} (u-t)^{-\gamma-1} \,du
	\right|  dt
	\\
	\begin{aligned}
		&\le
		\int_s^\vS \vS^{-\alpha} (\vS-t)^{-\gamma-1} \, dt +
		|\alpha| \,  \int_s^\vS u^{-\alpha-1}
		\int_s^u (u-t)^{-\gamma-1} \, dt \,du
		\\ &=
		\frac{\vS^{-\alpha} (\vS-s)^{-\gamma}}{-\gamma} +
		\frac{|\alpha|}{-\gamma}
		\int_s^\vS u^{-\alpha-1} (u-s)^{-\gamma} \, du
		< \infty .
	\end{aligned}
\end{multline*}
The product of bounded continuous function and integrable function is integrable.
Thus, $L(\vS,t)$ satisfies \eqref{eq:abocon}.
The theorem is proved.
\end{proof}

\subsubsection*{Integrability
with respect to the process \boldmath$X$}
For fixed $v>0$, the function
kernel $L(v,t)$ defined in
Theorem~\ref{thm:Lker-suf}
is continuous in $t$ in the interval
$(0,v)$.
With \cite[Lemma~4]{Part 1},
the asymptotics of $L(v,t)$ as
$t \to 0$ can be established:
\begin{gather*}
L(v,t) = O(t^{-\beta} v^{-\alpha-\gamma-1})
  \quad
\mbox{if $\alpha+\gamma<-1$},\\
L(v,t) \sim \frac{\alpha}{\mathrm{B}(-\gamma,\gamma{+}1)} \,
t^{-\beta} \ln(v/t)
\quad \mbox{if $\alpha+\gamma=-1$},\\
L(v,t) \sim
\frac{\alpha\, \mathrm{B}(-\gamma,\alpha{+}\gamma{+}1)}
{\mathrm{B}(-\gamma,\gamma{+}1)} \,
t^{-\alpha-\beta-\gamma-1}
\quad \mbox{if $\alpha+\gamma>-1$}.
\end{gather*}
The asymptotics in the other endpoint
is
\[
L(v,t) \sim \frac{t^{-\beta} v^{-\alpha}
(v-t)^{-\gamma-1}}
{\mathrm{B}(-\gamma, \gamma{+}1)}
\quad
\mbox{as}
\quad t\to v{-}\,.
\]
In order for $L(t,v)$ to be defined for all $t\in(0,T]$
and thus the relation ``$L(v, \mathbin{\cdot}) \in
\mathcal{E}_{p,k}$'' make sense, assume
$L(v,t) = 0$ for $0<v\le t$.
The relation $L(v, \mathbin{\cdot}) \in
\mathcal{E}_{p,k}$ holds true
if and only if
$\frac{1}{p} > \max(\gamma{+}1,\:
\beta{-}k,\:
\alpha{+}\beta{+}\gamma{+}1{-}k)$.

Let $\alpha$, $\beta$ and $\gamma$
satisfy \eqref{eq:XpowerVcond}
as well as $\gamma<0$.
We chose $p$ and $k$ from the
double inequalities
\begin{equation}\label{eq:selpk}
\begin{gathered}
\max(0, \: \gamma+1)  < \frac{1}{p} <
\min\!\left(1, \: \gamma+{\textstyle\frac32}\right), \\
\max(0, \: \beta, \: \alpha+\beta+\gamma+1)
< \frac{1}{p}+k <
\alpha+\beta+\gamma+\frac32.
\end{gathered}
\end{equation}
These inequalities are compatible:
the desired $p$ and $k$ exist.

Then $L(v,\mathbin{\cdot})
\in \mathcal{E}_{p,k}$
for all $v\in(0,T]$
and all functions within $\mathcal{E}_{p,k}$
are integrable with respect to  $X_t$
according to  Definition~\ref{def-PLI-3p}.
Then, with \eqref{eqLinIEQ-power},
\[
\int_0^v L(v,t) \, dX_t
=
\int_0^v s^\alpha \int_s^v
L(v,t) t^\beta (t-s)^\gamma \, dt \, dW_s
= \int_0^v dW_s = W_v.
\]
 

As a  corollary, we immediately obtain the main result of this section.
\begin{theorem}
	Let $\alpha\mathbin{>}-\frac12$,
	${-}1\mathbin{<}\gamma\mathbin{<}0$
	and $\alpha+\beta+\gamma\mathbin{>}{-}\frac32$.
	Then
	\begin{enumerate}
		 
			 
		\item
			The process $X$ that is well-defined by
			\eqref{eq:XpowerV},
			according to \cite[Theorem~1]{Part 1}, admits the inverse representation of the form
			\begin{align}
				\hspace{-20.5pt}W_v &= \int_0^v L(v,t)\, dX_t
				\nonumber \\ \hspace{-6.3pt}&=
				\frac{1}{\mathrm{B}(-\gamma, \gamma{+}1)}
				\int_0^v \! t^{-\beta}
				\left(\!\vS^{-\alpha} (\vS{-}t)^{-\gamma-1}
				+ \alpha \int_t^\vS \! u^{-\alpha-1}
				(u{-}t)^{-\gamma-1} \,du \!\right)
				dX_t .
				\label{eq:invRepgneg}
			\end{align}
		\item
			The integration in \eqref{eq:invRepgneg}
			can be formally performed according
			to \eqref{eq:defLPI3Kernel}, there exist $p$ and $k$, more precisely,   $p$ and $k$
			that satisfy \eqref{eq:selpk},
			for which $L(v,{{}\mathbin{\cdot}{}})  \in
			\mathcal{E}_{p,k}$, and integration
			in \eqref{eq:invRepgneg} can be performed
			according to Definition~\ref{def-PLI-3p}.
			
	\end{enumerate}
\end{theorem}

Note that justification of such an integration rule
			is given in Remark~\ref{rem:just-def-PLI-3p}.

\subsection[Case $\gamma\ge 0$ integer]
{Case \boldmath$\gamma\ge 0$ integer}
\label{ss:invrepgammainteger}
Let $\alpha>-\frac12$, $\gamma\in\mathbb{N}\cup\{0\}$
and $\alpha+\beta+\gamma\mathbin{>}{-}\frac32$.
According to Corollary~\ref{corollary:dXdt},
the process $X$ has a modification
which is $\gamma{+}1$ times
continuously differentiable.
For this modification,
\begin{equation}\label{eq:t-bXtdt}
	\frac{1}{t^\beta} \frac{d X_t}{dt} = t^{-\beta} \dotX_t
	= \int_0^t s^\alpha (t-s)^\gamma \, dW_s .
\end{equation}

If $\gamma=0$, then
\[
	t^{-\beta} \dotX_t = \int_0^t s^\alpha \, dW_s, \qquad
	W_t = \int_0^t s^{-\alpha} \, d(s^{-\beta} \dotX_s)
\]
is a representation of $W$.

If $\gamma\in\mathbb{N}$, then
$\int_0^t s^\alpha (t-s)^\gamma \, dW_s$
is a $\gamma$th antiderivative
of $\gamma! \, \int_0^t s^\alpha \, dW_s$.
This follows by induction from the representation
\[
	\int_0^v \int_0^t \! s^\alpha (t{-}s)^k \, dW_s \, dt
	=
	\int_0^v \! s^\alpha \! \int_s^v (t{-}s)^k \, dt \, dW_s
	=
	\frac{1}{k{+}1} \int_0^v \! s^\alpha (v{-}s)^{k+1} \, dW_s,
\]
$k\mathbin{>}{-}\frac12$. Thus,
\[
	\int_0^t s^\alpha \, dW_s = \frac{1}{\gamma!}
	\frac{d^\gamma}{dt^\gamma}
	\biggl(
	\int_0^t s^\alpha (t-s)^\gamma \, dW_s
	\biggr)
	= \frac{1}{\gamma!} \frac{d^\gamma (t^{-\beta} \dotX_t)}
	{dt^\gamma} .
\]
The representation of $W$ is
\[
	 W_t = \frac{1}{\gamma!}
	\int_0^t s^{-\alpha} \,
	d\!\left(
	\frac{d^\gamma (s^{-\beta} \dotX_s)}
	{ds^\gamma}
	\right).
\]

\subsection[Case $\gamma> 0$ non-integer]
{Case \boldmath$\gamma> 0$ non-integer}
\label{ss:invrepgammapositive}
Let $\alpha\mathbin{>}{-}\frac12$, $\gamma\mathbin{>}0$
and $\alpha+\beta+\gamma\mathbin{>}{-}\frac32$.
According to Corollary~\ref{corollary:dXdt},
the process $X$ has a modification
which is $\lceil \gamma{+}\frac12\rceil$ times
continuously differentiable.
For this modification, \eqref{eq:t-bXtdt}
holds true.

 
The process $\int_0^t s^\alpha \, dW_s$ is
continuous on $[0,T]$.

The process $\int_0^t \, s^\alpha (t-s)^\gamma \, dW_s$
is a $\gamma$th order fractional antiderivative
of the process
$\Gamma(\gamma{+}1) \int_0^t \, s^\alpha \, dW_s$.
Indeed,
\begin{multline*}
	\mathcal{I}^\alpha_{0+}\!\left(
	\int_0^t s^\alpha \, dW_s\right)
	=
	\frac{1}{\Gamma(\gamma)}
	\int_0^t u^{\gamma-1} \int_0^u s^\alpha \, dW_s \, du
	\\=
	\frac{1}{\Gamma(\gamma)}
	\int_0^t s^\alpha \int_u^t u^{\gamma-1} \, du \, dW_s
	=
	\frac{1}{\Gamma(\gamma{+}1)}
	\int_0^t s^\alpha (t-s)^\gamma \, dW_s .
\end{multline*}
Thus,
\[
	\int_0^t s^\alpha \, dW_s
	= \frac{1}{\Gamma(\gamma{+}1)}
	\, \mathcal{D}^{\gamma}_{0+} \!
	\biggl( \int_0^t s^\alpha (t-s)^\gamma \, dW_s \biggr)
	= \frac{1}{\Gamma(\gamma{+}1)}
	\, \mathcal{D}^{\gamma}_{0+} (t^{-\beta} \, \dotX_t) .
\]

If $\gamma$ is not integer, then $\gamma = \lceil\gamma\rceil - 1 + \{\gamma\}$
for $\lceil\gamma\rceil\in\mathbb{N}$ and $0<\{\gamma\}<1$.
Then
\begin{gather*}
	\begin{aligned}
	\mathcal{D}_{0+}^\gamma (t^{-\beta} \dotX_t)
		&=
	\frac{d^{\lceil\gamma\rceil}}
	    {dt^{\lceil\gamma\rceil}}
	\bigl( I^{1-\{\gamma\}}_{0+} (t^{-\beta} \dotX_t) \bigr)
		\\&=
	\frac{1}{\Gamma(1{-}\{\gamma\})} \,
	\frac{d^{\lceil\gamma\rceil}}
	    {dt^{\lceil\gamma\rceil}} \!
	\biggl(\int_0^t s^{-\beta} (t-s)^{-\{\gamma\}} \dotX_t \, ds \biggr),
	\end{aligned}\\
\int_0^t s^\alpha \, dW_s =
	\frac{1}{\Gamma(\gamma+1)\,\Gamma(1{-}\{\gamma\})} \,
	\frac{d^{\lceil\gamma\rceil}}
	    {dt^{\lceil\gamma\rceil}} \!
	\biggl(\int_0^t s^{-\beta} (t-s)^{-\{\gamma\}} \dotX_t \, ds \biggr),
	\\
	W_v =
	\frac{1}{\Gamma(\gamma+1)\,\Gamma(1{-}\{\gamma\})}
	\int_0^v
	t^{-\alpha}
	\, d\!\left(
	\frac{d^{\lceil\gamma\rceil}}
	    {dt^{\lceil\gamma\rceil}} \!
	\biggl(\int_0^t s^{-\beta} (t-s)^{-\{\gamma\}} \dotX_t \, ds \biggr)
	\right) .
\end{gather*}
Thus, we have constructed the representation for $W$.
	





%\section{Asymptotic growth at infinity}
%Consider \cite[Theorem~2.6]{DKM}.
%Let
%\begin{gather*}
%	m_k = \sup_{t\in[b_k,b_{k+1}]} (\ME |X(t)|^2)^{1/2}
%	\le C b_{k+1}^\rho ,
%	\\
%	\sup_{t,s\in[b_k,b_{k+1}]} (\ME|X(t) - X(s)|^2)^{1/2}
%	\le C b_{k+1}^r h^{\boldsymbol{\beta}},
%	\qquad 0\mathbin{<}\boldsymbol{\beta}\mathbin{<}1,
%	\quad |t-s|\mathbin{\le}h.
%\end{gather*}
%Put
%$b_k = e^k$ and $a(t)=t^\kappa \log^p(t)$, $p\mathbin{>}1$.
%Then condition (i) of Theorem~2.6 holds true.
%Condition (ii):
%\[
%	\sum_{k=0}^\infty \frac{m_k}{a_k}
%	\le
%	C \sum_{k=0}^\infty \frac{e^{(k+1)\rho}}
%	{e^{k \kappa} k^p}
%	< \infty
%	\quad \mbox{if} \quad
%	\rho \mathbin{\le} \kappa.
%\]
%Condition (iii):
%\begin{align*}
%	\sum_{k=0}^\infty
%	\frac{m_k^{1 - \frac{\boldsymbol{\gamma}}{\boldsymbol{\beta}}}
%	(b_{k+1} - b_k)^{\boldsymbol\gamma}
%	c_k^{\boldsymbol{\gamma}/\boldsymbol{\beta}}}
%	{a_k}
%	&\le
%	C \sum_{k=0}^\infty
%	\frac{e^{k \rho \, \left( 1
%	- \frac{\boldsymbol{\gamma}}{\boldsymbol{\beta}} \right)}
%	e^{k \boldsymbol{\gamma}} e^{k r \boldsymbol{\gamma} /
%	\boldsymbol{\beta}}}
%	{e^{k\kappa} k^\rho}
%	\le \\ &\le
%	C \sum_{k=0}^\infty
%	\frac{\exp\left\{k \left( \rho \left(
%	1 - \frac{\boldsymbol{\gamma}}{\boldsymbol{\beta}} \right)
%	+ \boldsymbol{\gamma} + \frac{r \boldsymbol{\gamma}}
%	{\beta} - \kappa \right) \right\}}
%	{k \rho}
%	< \infty
%\end{align*}
%if $\rho - \kappa + \boldsymbol{\gamma}
%\left( 1 + \frac{r}{\boldsymbol{\beta}} - \frac{\rho}{\boldsymbol{\beta}}
%\right) \le 0$
%for some $\boldsymbol{\gamma}\in(0,1]$.
%
%It can be under the following conditions.
%\begin{enumerate}
%	\def\theenumi{\arabic{enumi})}
%	\def\labelenumi{\theenumi}
%	\item $\kappa>\rho$.
%		In this case, regardless of the sign of
%		$1 + (r-\rho)/{\boldsymbol{\beta}}$,
%		we can choose $\boldsymbol{\gamma}>0$
%		sufficiently small to supply
%		$\rho - \kappa + \boldsymbol{\gamma}
%		\left( 1 + \frac{r-\rho}{\boldsymbol{\beta}}\right)
%		< 0$.
%	\item \label{item:2-krb-r0}
%		$\kappa = \rho = \boldsymbol{\beta}$ and $r=0$.
%	\item $\kappa = \rho$ and
%		$1 + \frac{r-\rho}{\boldsymbol{\beta}} \le 0$.
%		(Case \ref{item:2-krb-r0} is included here.)
%	\item $\rho > \kappa$,
%		$1 + \cfrac{r-\rho}{\boldsymbol{\beta}} < 0$ and
%		$0 < \cfrac{\rho - \kappa}{\frac{\rho-r}{\boldsymbol{\beta}}
%		- 1} \le 1$.
%		In this case we can choose
%		\[
%			1 \ge \gamma \ge \frac{\rho-\kappa}
%			{\frac{\rho-r}{\boldsymbol{\beta}} - 1} .
%		\]
%\end{enumerate}
%
%If to summarise and apply \cite[Corollary~2.10]{DKM},
%we get that under assumptions
%\begin{enumerate}
%	\def\labelenumi{(\roman{enumi})}
%	\item $\displaystyle
%		\sup_{s\le t} (\ME |X_s|^2)^{1/2} \le C t^\rho$,
%	\item $\displaystyle
%		\sup_{s,u\le t} (\ME |X_t - X_s|^2)^{1/2} \le
%		C t^r h^{\boldsymbol\beta}$,\quad
%		$0\mathbin{<}\boldsymbol{\beta}\mathbin{<}1$,\quad
%		$|s - u| \boldsymbol{\le} h$
%\end{enumerate}
%we have that
%\[
%	|X_t| \le t^{\kappa} \log^p t \xi,
%	\qquad
%	p\mathbin{>}1,
%\]
%where we can put $\kappa = \rho$ if
%$1 + (r-\rho)/{\boldsymbol{\beta}} \le 0$,
%and we can take any $\kappa>\rho$ and $\xi = \ldots\ldots\ldots$
%
%Consider
%\[
%\mathbb{E}
%=
%\ME|X_{t_2} - X_{t_1}|^2
%=
%2 \int_{t_1}^{t_2} u^\beta
%\int_u^{t_2} v^\beta
%\int_0^u s^{2\alpha} (u-s)^\gamma (v-s)^\gamma
%\, ds \, dv \, du .
%\]
%Here $\alpha>-\frac12$, $\gamma>-1$ and $\alpha+\beta+\gamma>-\frac32$.
%
%\paragraph{(a)}
%Let $\gamma<0$.
%Then $(v-s)^\gamma \le (v-u)^\gamma$, and
%\begin{align*}
%	\mathbb{E}
%	&=
%	2 C \int_{t_1}^{t_2} u^\beta
%	\int_u^{t_2} v^\beta (v-u)^\gamma u^{2\alpha+\gamma+1} \, dv \, du
%	= \\ &=
%	C \int_{t_1}^{t_2} u^{2\alpha + \beta + \gamma + 1}
%	\int_u^{t_2} v^\beta (v-u)^\gamma \, du \, dv
%	= \\ &=
%	C \int_{t_1}^{t_2} v^\beta
%	\int_0^v u^{2\alpha+\beta+\gamma+1}
%	(v-u)^\gamma \, du \, dv
%	= \\ &=
%	C \int_{t_1}^{t_2} v^{2\alpha + 2\beta + 2\gamma + 2} \, dv
%	=
%	C \, (t_2^{2\alpha + 2\beta + 2\gamma + 3} -
%	      t_1^{2\alpha + 2\beta + 2\gamma + 3}) .
%\end{align*}
%\paragraph{(aa)}
%Let $2\alpha + 2\beta + 2\gamma + 3 \in (0,1]$ and $\gamma<0$.
%Then
%$\mathbb{E} \le C \, (t_2 - t_1)^{2\alpha + 2\beta + 2\gamma +3}.$
%It means that
%$r=0$,
%$\boldsymbol{\beta} = \alpha + \beta + \gamma + \frac32 = \rho$,
%and we can put $\kappa = \rho$,
%\[
%	a(t) = t^{\alpha + \beta + \gamma + \frac32} \log^p t,
%	\qquad
%	\forall p>0.
%\]
%\paragraph{(ab)}
%Let $2\alpha + 2\beta + 2\gamma + 3 > 1$ and $\gamma < 0$.
%Then
%\[
%	t_2^{2\alpha+2\beta+2\gamma+3} - t_1^{2\alpha+2\beta+2\gamma+3}
%	\le
%	C t_2^{2\alpha+2\beta+2\gamma+1} (t_2-t_1) .
%\]
%It means that $r=\alpha+\beta+\gamma+1>0$,
%$\boldsymbol{\beta} = \frac12$,
%$\rho = \alpha+\beta+\gamma+\frac32$,
%and we can take any $\kappa > \alpha+\beta+\gamma+\frac32$.
%Note that
%\[
%	1 + \frac{r-\rho}{\boldsymbol{\beta}}
%	=
%	1 + \frac{\alpha + \beta + \gamma + 1 -
%	(\alpha + \beta + \gamma + \frac32)}
%	{1/2} = 0,
%\]
%therefore we can take $\kappa = \alpha + \beta + \gamma + \frac32$.
%
%\paragraph{(b)}
%Let $\gamma=0$,
%Assume that $\beta>-1$.
%\begin{align*}
%	\mathbb{E}
%	&=
%	\frac{2}{2\alpha+1}
%	\int_{t_1}^{t_2} u^\beta
%	\int_u^{t_2} v^\beta u^{2\alpha+1} \, dv \, du
%	= \\ &=
%	C \int_{t_1}^{t_2} u^{2\alpha+\beta+1}
%	(t_2^{\beta+1} - u^{\beta+1}) \, du
%	\le \\ &\le
%	C \, (t_2^{\beta+1} -  t_1^{\beta+1}) \,
%	(t_2^{2\alpha+\beta+2} - t_1^{2\alpha+\beta+2})
%	\le \\ &\le
%	\begin{cases}
%		C \, (t_2-t_1)^{2\alpha+2\beta+3}
%		& \mbox{if $\beta\mathbin{\le}0$ and
%		$0\mathbin{<}2\alpha+\beta+2\mathbin{\le}1$},
%		\\
%		C \, t_2^\beta (t_2-t_1)^{2\alpha+\beta+3}
%		& \mbox{if $\beta\mathbin{>}0$ and
%		$0\mathbin{<}2\alpha+\beta+2\mathbin{\le}1$},
%		\\
%		C \, (t_2-t_1)^{\beta+2} t_2^{2\alpha+\beta+1}
%		& \mbox{if $\beta\mathbin{\le}0$ and
%		$2\alpha+\beta+2\mathbin{>}1$},
%		\\
%		C \, t_2^{2\alpha+2\beta+1} (t_2-t_1)^2
%		& \mbox{if $\beta\mathbin{>}0$ and
%		$2\alpha+\beta+2\mathbin{>}1$}
%	\end{cases}
%	\\ &=: \mathbb{E}_0.
%\end{align*}
%\paragraph{(c)}
%Let $\gamma>0$.
%Then
%\[
%	\mathbb{E} \le (t_2-t_1)^{2\gamma} \mathbb{E}_0,
%\]
%where $\mathbb{E}_0$ comes from case \textbf{(b)}.


\begin{thebibliography}{8}


\bibitem{ASVY2014}
   E.~Azmoodeh, T.~Sottinen, L.~Viitasaari, A.~Yazigi.
   Necessary and sufficient conditions for {H\"older} continuity of
   {Gaussian} process.
   \textit{Statist. Probab. Lett.}, 2014, Vol.~94, 230--235.

\bibitem{Biagini2008}	
    F.~Biagini, Y.~Hu, B.~{\O}ksendal, T.~Zhang.
    \textit{Stochastic Calculus for Fractional {Brownian} motion
	and Applications}.
	London : Springer, 2008.




	\bibitem{DKM}
		M.~Dozzi, Yu.~Kozachenko, Yu.~Mishura, K.~Ralchenko.
		Asymptotic growth of trajectories of multifractional
		Brownian motion, with statistical applications to
		drift parameter estimation.
		\textit{Stat. Interference Stoch. Process.},
		2018, Vol.~21, 21--52.

\bibitem{LDecr2005}
L.~Decreusefond.
Stochastic integration with respect to Volterra processes.
\textit{Ann. I. H. Poincar\'e (B) Probab. Statist.}, 2005, Vol.~41, No.~2, 123--149.
	\bibitem{Koz2015}
		Y.~Kozachenko, A.~Melnikov and Y.~Mishura.
		On drift parameter estimation in models with fractional Brownian motion.
		\textit{Statistics}, 2015, Vol.~49, No.~1, 35--62.

	\bibitem{MishSh2017}
		Yu.~Mishura, G.~Shevchenko.
		\textit{Theory and Statistical Applications of Stochastic Processes}.
		ISTE, London, 2017; Wiley, Hoboken, 2017.

\bibitem{arXiv03405}
Yu.~Mishura, G.~Shevchenko, S.~Shklyar.
Gaussian processes with Volterra kernels.
arxiv:\texttt{2001.03405}

	\bibitem{Part 1}
		Yu.~Mishura and S.~Shklyar.
		Gaussian Volterra processes with power-type kernels.
		Submitted to \textit{Modern Stochastics: Theory and Applications}.
		
	\bibitem{Norros1999}
		I.~Norros, E.~Valkeila and J.~Virtamo.
		An elementary approach to a Girsanov formula and
		other analytical results on fractional Brownian motions.
		\textit{Bernoulli}, 1999, Vol.~5, No.~4, 571--587.

	\bibitem{Nualart2006}
		D.~Nualart.
		Stochastic calculus with respect to
		fractional Brownian motion.
		\textit{Annales de la facult\'e de
		Toulouse Math\'ematiques},
		2006, Vol.~15, No.~1, \mbox{63--77}.



	\bibitem{PapirasTaqqu2001}
		V.~Papirasand M.~S.~Taqqu.
		Are classes of deterministic integrands
		for fractional Brownian motion
		of an interval complete?
		\textit{Bernoulli}, 2001, Vol.~7, No.~6, 873--897.

\end{thebibliography}

\pagebreak

\section*{Scraps}
\section{The inverse representation for $\gamma>0$}
This section has been rewritten as Subsections
\ref{ss:invrepgammainteger} and \ref{ss:invrepgammapositive}.

Let $\gamma>0$.
In the formula \eqref{eq:XpowerV}
change the order of integration:
\[
X_t = \int_0^t u^\beta \int_0^u s^\alpha \, (u-s)^\gamma \, dW_s\, du.
\]
The process $\int_0^u s^\alpha \, (u-s)^\gamma \, dW_s$ is
well-defined. As a result, $X_t$ is differentiable and its
derivative is equal to
\[
X'_t = t^\beta \int_0^t s^\alpha \, (t-s)^\gamma \, dW_s.
\]
Denote
\[
\phi_u = \int_0^u s^\alpha \, dW_s; \qquad
U_t = t^{-\beta} X'_t .
\]
Then
\begin{gather*}
X'_t = t^\beta \int_0^t (t-s)^\gamma \, d\phi_s, \\
U_t = \int_0^t (t-s)^\gamma \, d\phi_s .
\end{gather*}
Perform integration by parts:
\begin{align}
U_t &= (t-s)^\gamma \phi_s \biggr| _0^t - \int_0^t \psi_s d((t-s)^\gamma)
= \gamma \int_0^t \psi_s (t-s)^{\gamma-1} \, ds .
\label{eq:ieqUphi}
\end{align}


If $0<\gamma<1$, then the solution to the integral equation \eqref{eq:ieqUphi} is
\[
\psi_s = \frac{1} {\gamma \, \mathrm{B} (\gamma,\, 1{-}\gamma)} \,
\frac{d}{ds} \int_0^s \frac{U_t \, dt}{(s-t)^{-\gamma}} .
\]
If $\gamma$ is a positive integer, then \eqref{eq:ieqUphi} becomes
\[
U_t = \gamma \int_0^t \psi_s (t-s)^{\gamma-1} \, ds
= \gamma! \int_0^t \int_0^{s_{\gamma-1}} \!\! \dots
\int_0^{s_2} \int_0^{s_1} \phi(s_0) \, ds_0 \, ds_1 \ldots ds_{\gamma-2} \, ds_{\gamma-1},
\]
whence the solution is
\[
\psi_s = \frac{1}{\gamma!} \, \frac{d^\gamma}{ds^\gamma} U_s .
\]
If $\gamma = k + \delta$ with $0 < \delta <1$, then
\begin{align*}
\int_0^u (u-t)^{-\delta} U_t \, dt &= \gamma \int_0^u (u-t)^{-\delta} \int_0^t \psi_s (t-s)^{\gamma-1} \, ds\, dt
= \\ &= \gamma \int_0^u \int_s^u (u-t)^{-\delta} (t-s)^{\gamma-1} \, dt \, \psi_s \, ds
= \\ &= \gamma \, \mathrm{B}(1{-}\delta, \gamma) \, \int_0^u (u-s)^k \psi_s \, ds ,
\end{align*}
and the solution $\psi_s$ is
\[
\psi_s = \frac{1}{\Gamma(\gamma+1) \, \Gamma(1-\delta)}
\frac{d^{k+1}}{ds^{k+1}} \left(
\int_0^s (s-t)^{-\delta} U_t \, dt \right).
\]
Here we use the relation
$\gamma \, \mathrm{B}(1{-}\delta, \gamma) \, k! = \Gamma(\gamma+1) \, \Gamma(1-\delta)$.

Finally,
\[
W_t = \int_0^t s^{-\alpha} d\psi_s .
\]

\section{Stuff for the Introduction}
\subsection{Integration w.r.t.~the Gaussian process}
Citation of Nualart (2006).

In \cite{Nualart2006}, the function $\phi$
should be such that
$K^* \phi$ is well-defined and
$K^* \phi \in L^2[0,T]$.
There is a Hilbert (or scalar-product) space $\mathcal{H}$
of functions with norm $\|\phi\|_{\mathcal H} = \|K^* \phi\|_2$,
and, obviously, the $K^*$ is a linear
continuous operator $\mathcal{H} \to L^2[0,T]$.

We do that the other way.
We prove that the operator $K^*$ is
$\mathcal{E} \to L^2[0,T]$-continuous
for some Lp space $\mathcal{E}$.

\subsection{Inverse representation for
three-function Volterra Gaussian process}
Citation of Mishura, Shevchenko and Shklyar (arXiv:\texttt{2001.03405}).

The solution to equation
\[
\int_s^\vS L(\vS,t) a(s) b(t) c(t-s) \, dt = 1
\]
is found in
Mishura \textit{et al}.\@{} \cite[Proposition 3]{arXiv03405}.
In that solution, they use the ``Sonine pair'' with the function $c(s)$, which is
a function $p(s)$ that satisfies the integral equation
\[
\int_0^t p(s) \, c(t-s) \, ds = 1, \qquad t\in(0,T].
\]
The ``Sonine pair'' with the power function $s^\gamma$ exists for all $\gamma\in(-1,0)$;
it is equal to $s^{-\gamma-1} / \, \mathrm{B}(-\gamma, \gamma{+}1)$.
\end{document}

% The one-paragraph proof of Corollary~\ref{corollary:ELbound}
% (Corollary 1 as of 2022-03-15).
% No need to keep it unless we are going to make the paper shorter.
The following corollary follows
from Proposition~\ref{prop:PW-L-cont}
applied for operator $K^*$ with parameters
$(\alpha, \beta{-}k, \gamma)$
and function $\phi(t) t^k$.
The norm of the operator $K^* : \mathcal{E}_{p,k} \to L^2[0,T]$
will be not more than $c(p; \alpha, \beta{-}k, \gamma)$,
where $c(p; \alpha, \beta, \gamma)$ is defined
in the proof of Proposition~\ref{prop:PW-L-cont}.

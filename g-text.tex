\documentclass{article}
\usepackage{amsmath}
\DeclareMathOperator{\ME}{\mathsf{E}}
\DeclareMathOperator{\var}{\mathsf{var}}
\usepackage{amsthm}
\theoremstyle{plain}
\newtheorem{proposition}{Proposition}
\theoremstyle{definition}
\newtheorem{defn}{Definition}
\begin{document}
\section{Motivation}
The fractional Brownian motion ---
a self-similar zero-mean Gaussian process with stationary increments
and power incremental variances $\ME (B^H_t - B^H_s)^2 = (t-s)^{2H}$
admits a Molchan representation on $[0,\infty)$:
\begin{gather*}
B^H_t = \int_0^t K(t,s) \, dW_s, \\
K(t,s) = c_H s^{0.5-H} \biggl( t^{H-0.5} (t{-}s)^{H-0.5} -
\left(H{-}{\textstyle\frac12}\right) \int_s^t u^{H-1.5} (u{-}s)^{H-0.5}\, du
\biggr)
\end{gather*}
if $0<H<1$,
\begin{gather*}
K(t,s) = 
\left(H{-}{\textstyle\frac12}\right) c_H
s^{0.5-H} \int_s^t u^{H-0.5} (u-s)^{H-1.5} \, du \quad
\mbox{if $\frac12{<}H{<}1$},
\\
c_H = \left( \frac
{2 H \, \Gamma(1.5-H)}
{\Gamma(H+0.5)\, \Gamma(2-2H)}
\right)^{1/2} .
\end{gather*}

\section{Generalization 1: Three functions}
Consider the kernel of the form
\[
K(t,s) = a(s) \int_s^t b(u) c(u-s) \, du
\]
Then the Volterra process is
\begin{equation}\label{eq:3Xabc}
X_t = \int_0^t a(s) \int_s^t b(u) c(u-s)\, du \, dW_s .
\end{equation}
\begin{proposition}\label{prop:1}
Let $a\in L^p[0,T]$, \, $b\in L^q[0,T]$
and $c\in L^r[0,T]$,
\[
p\ge 2,\quad
q\ge 1,\quad
r\ge 1,\quad
\frac{1}{p}+\frac{1}{q}+\frac{1}{r}\le \frac32.
\]
Then the process $X$ is well-defined on $[0,T]$ by \eqref{eq:3Xabc}.
If, in addition  $\frac{1}{p}+\frac{1}{r} < \frac32$,
then the process $X$ is  path-continuous (up to modification).
If $q>1$ and $\frac{1}{p}+\frac{1}{q}+\frac{1}{r} < \frac32$,
then the process $X$ is H\"older-continuous up
to order $\frac32 - 1/q - \max(\frac12, \, 1/p+1/r)$.
\end{proposition}
Fractional Brownian motion with Hurst index $H>0.5$
is a particular case of process \eqref{eq:3Xabc}.
However, Proposition~\ref{prop:1}
underestimates the order in H\"older condition for the fBm.

\section{Generalization 2: Three power functions}
Consider the case where $a$, $b$ and $c$ are power functions.
\[
X_t = \int_0^t s^\beta \int_s^t u^\beta (u-s)^\gamma \,  du \, dW_t.
\]
The process $X_t$ is well-defined if
\[
\alpha>-\frac12, \quad \gamma>-1, \quad
\mbox{and} \quad
\alpha+\beta+\gamma>-\frac32.
\]

The process $X$ is self-similar with exponent
\[
H = \alpha+\beta+\gamma+\frac32 .
\]

The process $X$ satisfies the H\"older condition
up to order $\min(\gamma+\frac32, \, 1)$
on the interval $[t_0, T]$,
and it satisfies 
the H\"older condition
up to order $\min(H, \gamma+\frac32, \, 1)$
on the interval $[0, T]$.
Here $0<t_0<T$.

\subsection{Incremental variance and generalized quasi-helix condition}
\begin{proposition}\label{prop:localvar}
Let $t_0 > 0$.
The asymptotics of the incremental variance as $t\to t_0$
is as follows:
\begin{equation*}
% \refstepcounter{equation}
% \label{eq:asymp-localvar-1}
\begin{array}{l@{\qquad}l}
\displaystyle \var(X_{t}-X_{t_0}) \sim \frac
{t_0^{2\alpha+2\beta} \, |t - t_0|^{2\gamma+3} \,
\mathrm{B}(\gamma{+}1,\, {-}2\gamma{-}1)} {(\gamma + 1)\,(2\gamma
+ 3)}
& \mbox{if $\gamma<-\frac12$,}
\quad (\theequation) \\
\displaystyle \var(X_{t}-X_{t_0}) \sim t_0^{2\alpha+2\beta} \,
(t - t_0)^2 \, \ln\!\left( \frac{t_0}{|t - t_0|} \right)
& \mbox{if $\gamma=-\frac12$,} \\
\displaystyle \var(X_{t}-X_{t_0}) \sim
t_0^{2\alpha+2\beta+2\gamma+1} \, (t - t_0)^2 \,
\mathrm{B}(2\alpha{+}1,\, 2\gamma{+}1) & \mbox{if
$\gamma>-\frac12$.}
\end{array}
\end{equation*}
\end{proposition}


\begin{defn}
Process $\{X_t, \; t\mathbin{\in}[t_0,T]\}$
is a generalized  quasi-helix  with
exponents $\lambda_i>0, i=1,2$   if there exist two constants $C_i>0, i=1,2$ such that for any
$t_0  \le  t_1 < t_2  \le  T$
\[
C_1 (t_2-t_1)^{2\lambda_1} \le \var(X_{t_2} - X_{t_1}) \le
C_2 (t_2-t_1)^{2\lambda_2}.
\]
\end{defn}

\vspace{3pt}{\def\arraystretch{1.3}
The exponents in the generalized quasi-helix condition:\\
\begin{tabular}{|p{0.35\linewidth}|p{0.27\linewidth}|p{0.2\linewidth}|}
\hline
& \multicolumn{2}{p{0.50\linewidth}|}{the exponents in the generalized  quasi-helix condition}\\
& on the interval $[0,T]$ are
& on any interval $[t_0,T]$ are\\
 \hline
If $\gamma\mathbin{<}-\frac12$ and $\alpha+\beta\le 0$,
& $\gamma{+}\frac32$ and $\alpha{+}\beta{+}\gamma{+}\frac32$
& $\gamma{+}\frac32$ \\ \hline
If $\gamma\mathbin{<}-\frac12$ and $\alpha+\beta\ge 0$,
& $\alpha{+}\beta{+}\gamma{+}\frac32$ and $\gamma{+}\frac32$
& $\gamma{+}\frac32$ \\ \hline
If $\gamma\mathbin{=}-\frac12$ and $\alpha+\beta< 0$,
& $1$ and $\alpha+\beta+1$
& $1$ and $1{-}\epsilon$ \\ \hline
If $\gamma\mathbin{=}-\frac12$ and $\alpha+\beta \ge 0$,
& $\alpha{+}\beta{+}1$ and $1{-}\epsilon$
& $1$ and $1{-}\epsilon$ \\  \hline
If $\gamma\mathbin{>}-\frac12$ and $\alpha{+}\beta{+}\gamma \mathbin{\le} -\frac12$,
& $1$ and $\alpha{+}\beta{+}\gamma{+}\frac32$
& $1$\\  \hline
If $\gamma\mathbin{>}-\frac12$ and $\alpha{+}\beta{+}\gamma \mathbin{\ge} -\frac12$,
& $\alpha{+}\beta{+}\gamma{+}\frac32$ and $1$
& $1$\\
\hline
\end{tabular}}

\end{document}

